\section{Restrictions}
\label{sec:restrictions}

From the requirements in \cref{sub:summary}, a requirement of being able to control what music is played at the venue were stated. An possible concept for solving the this problem is by the use of putting restrictions on what tracks can be added to the playlist. This section will present the the concepts of white- and black-listing.

\paragraph{Whitelisting} is a concept of being able to choose what entities is a accepted as a valid input. Strictly if $A$ is the set of accepted inputs, $B$ is the whitelist set and $i$ is the input then $i \in A$ \textbf{if and only if} $i \in B$.

\paragraph{Blacklisting} is the "opposite" in a sense, the concept is that the blacklist excludes specific entities from being a valid input. Entities in the blacklist set is excluded from the set of accepted inputs. Strictly $A$ is the set of accepted inputs, $B$ is the blacklist set and $i$ is the input set then $i \in A$ \textbf{if and only if} $i \notin B$.

With these concepts it is possible to restrict the music catalogue to a subset of allowed tracks, satifying the requirement of being able to control what music is being played. Restrictions can also be improving a sense of continuity between tracks if the restrictions limits allowed tracks to certain genre or mood, although this should be just be suplement to another concept, analysing the tracks on the playlist.

Futher more the owner of Fabrikken\cref{sec:fabrikken} wanting to be able to have timed intervals of restrictions, so that the restrictions acts accordingly to the progession in intensity and mood of the bar environment. This is quite trivialy implemented with a selective controlstructure, checking if the restriction is relevant at that time of day.