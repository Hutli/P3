\section{Restrictions}
\label{sec:restrictions}

As described in \cref{FACTOR}, and confirmed by with our industry partner in \cref{sec:fabrikken}, the system should have the ability to control what guests can vote for. The concept described for solving this problem is by putting restrictions on what tracks can be added to the playlist. This section will present how this concept is designed, by white- and blacklisting.

\paragraph{Whitelisting} is a concept of being able to choose what entities are accepted as a valid input. Strictly if $A$ is the set of accepted inputs, $B$ is the whitelist set and $i$ is the input then $i \in A$ \textbf{if and only if} $i \in B$. If there are multiple whitelists, the input is valid if it is allowed by any of them.
To get a better understanding of how this works mathematically, we can convert a whitelist, \emph{w}, to a logical expression evaluating to \textbf{true}, if the item is on the whitelist and \textbf{false} if not. With $n$ whitelists, these logical expressions can be combined into a logical expression evaluating to \textbf{true} if the track is allowed and \textbf{false} if not. This can be seen in \cref{eq:whitelist}. When $n$ = $0$, there are no whitelists, which tracks have to be on, and all tracks are therefore accepted. We therefore define expression to return \textbf{true} if $n$ = $0$.

\begin{equation}
\label{eq:whitelist}
	w_1 \vee w_2 \vee w_3 \vee \dots \vee w_n
\end{equation}

\paragraph{Blacklisting} is the opposite of whitelisting. The concept is that the blacklist excludes specific entities from being a valid input. Entities in the blacklist set are excluded from the set of accepted inputs. Strictly, if $A$ is the set of accepted inputs, $B$ is the blacklist set and $i$ is the input set, then $i \in A$ \textbf{if and only if} $i \notin B$.
We can, in the same way as with whitelists, convert a blacklist to a logical expression returning \textbf{true} if the item is on the blacklist or \textbf{false} if not, and again, with $n$ blacklists, create a combined logical expression that would evaluate to \textbf{true} if a track is allowed and \textbf{false} if not, as seen in \cref{eq:blacklist}. When $n$ = $0$, there are no blacklists. When there are no blacklists no tracks are filtered and all tracks return true with respect to the blacklist set. We therefore define the whole, including the negation, expression to return \textbf{true}.

\begin{equation}
\label{eq:blacklist}
	\neg(b_1 \vee b_2 \vee b_3 \vee \dots \vee b_n)
\end{equation}

For an entity to be accepted, it must be evaluated as \textbf{true} in both of the combined logical expressions in \cref{eq:blacklist} and \cref{eq:whitelist}. These expressions can therefore be combined into one expression, as seen in \cref{eq:blacklist&whitelist}. The logic of this combined expression can be stated as follows; \textbf{If and only if} a track is allowed by \textbf{any} of the whitelists, or there are no whitelists, \textbf{and} is \textbf{not} allowed by \textbf{any} of the blacklists, or there are no blacklists, the track is allowed.

\begin{equation}
\label{eq:blacklist&whitelist}
		(w_1 \vee w_2 \vee \dots \vee w_n) \wedge \neg(b_1 \vee b_2 \vee \dots \vee b_n)
\end{equation}

One restriction is a collection of rules on any of the meta data that a track contains. This means that if you do not want to hear \enquote{Still Alive} by \enquote{Lisa Miskovsky} this is a single blacklist restriction. People would still be able to hear other songs titled \enquote{Still Alive} or other music from \enquote{Lisa Miskovsky}. For a track to be allowed it must be meet all criteria on the set of restrictions. For \emph{n} whitelists and \emph{m} blacklists, with an arbitrary number of criteria, the complete and final expression would look as seen in \cref{eq:restrictions}.

\begin{eqnarray}
	\label{eq:restrictions}
	((w_{1.1} \wedge \dots \wedge w_{1.a}) \vee (w_{2.1} \wedge \dots \wedge w_{2.b}) \vee \dots \vee (w_{n.1} \wedge \dots \wedge w_{n.c})) \wedge \nonumber \\ \neg((b_{1.1} \wedge \dots \wedge b_{1.d}) \vee (b_{2.1} \wedge \dots \wedge b_{2.e}) \vee \dots \vee (b_{m.1} \wedge \dots \wedge b_{m.f}))
\end{eqnarray}

With these concepts, it is possible to restrict the music catalogue to a subset of allowed tracks, satisfying the requirement of being able to control what music is being played.

In \cref{alg:Search}, pseudo code for an implementation of the functionality, described earlier in this section, is shown. This algorithm uses the same logic as described in \cref{eq:blacklist&whitelist}. First, a list of tracks are found from a search query. For each of these tracks, a property, telling whether the track is allowed or not, is set by the \enquote{Restrict} function. This function evaluates each restriction, the predicates that return whether an item is on the list or not. For each of these restrictions, two disjunctions are formed from the whitelists and the blacklists. If any whitelists are found, an extra boolean is also set. In the last line, it then returns a conjunction from these two disjunctions, where the blacklist disjunction is negated, according to the logic described in \cref{eq:blacklist&whitelist}. The boolean for checking whether a whitelist exists is used, since the whitelist disjunction returns false, if there are no whitelists, and this conflicts with the definition of whitelists when $n$ = $0$. We can express that $WhitelistExists \Rightarrow IsOnWhitelist$, resulting in the expression always returning true when there are no whitelists. This is not necessary for blacklists since the negation results in the expression returning true if no blacklists exist.

\begin{algorithm}[htbp] \caption{Algorithm for filtering tracks in a search}\label{alg:Search}
\begin{algorithmic}[1]
	\Function{Search}{$query$, $restrictions$}
		\State{$results$ := $GetResults$($query$)}
		\ForAll{$track$ \textbf{in} $results$}
			\State{$track.IsAllowed$ := $Restrict$($track$,$restrictions$)}
		\EndFor{}
		\Return{$results$}
	\EndFunction{}\\
	\Function{Restrict}{$track$, $restrictions$}
		\State{$WhitelistExists$ := $false$}
		\State{$IsOnWhitelist$ := $false$}
		\State{$IsOnBlacklist$ := $false$}
		\ForAll{$restriction$ \textbf{in} $restrictions$}
			\If{$restriction.Type$ = $whitelist$}
				\State{$WhitelistExists$ := $true$}
				\State{$IsOnWhitelist$ := $IsOnWhitelist$ $\vee$ $restriction$($track$)}
			\ElsIf{$restriction.Type$ = $blacklist$}
				\State{$IsOnBlacklist$ := $IsOnBlacklist$ $\vee$ $restriction$($track$)}
			\EndIf{}
		\EndFor{}
		\State{\Return{($WhitelistExists$ $\to$ $IsOnWhitelist$) $\wedge$ $\neg IsOnBlacklist$}}
	\EndFunction{}
\end{algorithmic}
\end{algorithm}

Besides enabling administrators to restrict the search space, restrictions can also improve the music flow (described in \cref{sub:MusicFlow}). If the restrictions limits tracks to a certain genre or mood, within a certain timespan, users will only be able to vote for songs that fit the current music theme. This was also expressed by the owner of Fabrikken (see \cref{sec:fabrikken}). He expressed that he wanted to be able to have timed intervals for restrictions, so that the restrictions act accordingly to the progression of intensity and mood of the bar environment. This is quite trivially implemented with a selective control structure, checking if the restriction is relevant at that time of day. This solution to improve music flow should only be supplement to another concept, the system being able to analyse the tracks on the playlist and automatically choose the next track. An algorithm for this is presented in \cref{sec:algorithm}.