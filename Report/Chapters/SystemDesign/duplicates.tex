\subsection{Track duplicates in music catalog}
\label{sec:duplicates}

In an extensive music catalog, some entries may contain the exact same content, but have varying metadata related to it. This is a problem when working with the metadata as a tracks identifier. When voting on, this allows multiple instances of same track on the playlist at same time. When restricting tracks, the may allow some tracks to pass through the restrictions, this is equally a problem if the metadata is poorly formated or wrong.

To cope with some of these conflicts in related metadata, a generalisation is required. A common unique identifier as an when the track was made or the actual data of track, can beneficial in reducing the number of "duplicates", but when a conflict is found which versions of the metadata is left out from the searchspace, The most specific or the most general? The problem with chosing the most general is that, some vital information for a user might be lost, also the most specific might false. 