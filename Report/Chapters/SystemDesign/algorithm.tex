\section{Algorithm}
\label{sec:algorithm}

The music at a bar should never stop playing as described in
\cref{musthave}. Specifically this means that, when there is not voted
on any track on the current playlist, the system should not stop
playing. To handle this, an algorithm was designed to choose a track
similarly to the previously played tracks. In \cref{sub:MusicFlow}
it's described that the music may not suddenly change genre. The
algorithm described in \cref{alg:smartfind} should prevent sudden
changes in genre, by looking at the 9 last played track and finding a
similar tracks to these.

\makeatletter
% start with some helper code
% This is the vertical rule that is inserted
\newcommand*{\algrule}[1][\algorithmicindent]{\makebox[#1][l]{\hspace*{.5em}\vrule height .75\baselineskip depth .25\baselineskip}}%

\newcount\ALG@printindent@tempcnta
\def\ALG@printindent{%
    \ifnum \theALG@nested>0% is there anything to print
        \ifx\ALG@text\ALG@x@notext% is this an end group without any text?
            % do nothing
            \addvspace{-1pt}% FUDGE for cases where no text is shown, to make the rules line up
        \else
            \unskip
            % draw a rule for each indent level
            \ALG@printindent@tempcnta=1
            \loop
                \algrule[\csname ALG@ind@\the\ALG@printindent@tempcnta\endcsname]%
                \advance \ALG@printindent@tempcnta 1
            \ifnum \ALG@printindent@tempcnta<\numexpr\theALG@nested+1\relax% can't do <=, so add one to RHS and use < instead
            \repeat
        \fi
    \fi
    }%
% the following line injects our new indent handling code in place of the default spacing
\patchcmd{\ALG@doentity}{\noindent\hskip\ALG@tlm}{\ALG@printindent}{}{\errmessage{failed to patch}}
\makeatother
% end vertical rule patch for algorithmicx

\renewcommand{\algorithmicforall}{\textbf{for each}}

\begin{algorithm}[hbtp]
\caption{Algorithm for finding the next track to be played, if the playlist is empty.}\label{alg:smartfind}
\begin{algorithmic}[1]
	\Function{nextTrack}{$trackHistory$} \Comment{$trackHistory$ is a list of previously played tracks.}
		\State{$k = 9$}
		\State{$l = empty$ $list$}
		\If{$trackHistory.length < 9$}
			\State{$k = trackHistory.length$}
		\EndIf{}
		\For{$i = 0$ \textbf{to} $k$}
			\ForAll{$artist$ \textbf{in} $trackHistory[i].Artists$}
				\ForAll{$relatedArtist$ \textbf{in} $artist.relatedArtists$}
					\If{$l.contains(relatedArtist)$}
						\State{$relatedArtist.count += 1$}
					\Else{}
						\State{$l.add(relatedArtist)$}
						\State{$relatedArtist.count = 1$}
					\EndIf{}
				\EndFor{}
			\EndFor{}
		\EndFor{}
		\State{$x = null$}
		\ForAll{$artist$ \textbf{in} $l$}
			\If{$x = null$ || $artist.count > x.count$}
				\State{$x = artist$}
			\EndIf{}
		\EndFor{}
		\ForAll{$track$ \textbf{in} $x.top$}
			\If{$!trackHistory.contains(track)$}
			\State{}\Return{$track$}
			\EndIf{}
		\EndFor{}
	\EndFunction{}
\end{algorithmic}
\end{algorithm}

The users vote on tracks which forces the algorithm to chose a new song based on the previously played tracks, in order to match the music genre. Spotify does not contains meta data about which songs are related. Instead Spotify has related artist so the algorithm looks at the artists of the nine last played tracks at line 2 to 7. Theses artists has some related artists, which are iterated through at line 8. The most related artist among the nine tracks will be chosen. At line 15 to 17, the most related artist are chosen, and the first top tracks which haven't been played are chosen to be played next at line 18 to 20. In order to make the algorithm chose the highest rated track, the for each loop at line 18 has to run sequentially.
