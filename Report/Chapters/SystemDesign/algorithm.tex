\section{Next Track Based on History Algorithm}
\label{sec:algorithm}

The music at a bar should never stop playing as described in
\cref{sub:summary}. Specifically this means that, when there is not voted
on any track on the current playlist, the system should not stop
playing. In \cref{sub:MusicFlow} it's described that the music may not
suddenly change genre. To handle this, an algorithm was designed to
choose a track similarly to the previously played tracks. The
algorithm described as pseudo code in \cref{alg:smartfind} should
prevent sudden changes in genre.

As users vote on tracks, patterns in the music style
emerges. \Cref{alg:smartfind} uses this pattern to find the next track
to play. Spotify does not contains meta data about which songs are
related. Instead, Spotify has meta data about related artists. The algorithm finds the most appearing artist across the last nine played tracks. When this artist is found, one of the artists top tracks, which were not already played, is returned by the algorithm.

Lines 4 to 5 is a check in case the length of the history is not larger than $tracksToLookAt$. $tracksToLookAt$ is then set to the length of the history.

In lines 7 to 14, every artist in the nine last played tracks is iterated. For each of artist found, the artist's related artists, collected from Spotify meta data, is stored. For every related artist found, a check is made in lines 9 to 13. It is counted how many times each related artist appears.

In lines 15 to 18, the most related artist is determined, that is the related artist that was encountered the most in lines 6 to 13.

Now that the most related artist is found, its top tracks is iterated in lines 19 to 21. When a top track is found that has not already been played, the algorithm terminates.

\begin{algorithm}[hbtp]
\caption{Algorithm for finding the next track to be played, if the playlist is empty.}\label{alg:smartfind}
\begin{algorithmic}[1]
	\Function{nextTrack}{$trackHistory$} \Comment{$trackHistory$ is a list of previously played tracks.}
		\State{$tracksToLookAt = 9$}
		\State{$relatedArtists = empty list$}
		\If{$trackHistory.length < tracksTolookAt$}
			\State{$tracksToLookAt = trackHistory.length$}
		\EndIf{}
		\State{$lastPlayedTracks = trackHistory.getLast(tracksToLookAt)$}
		\ForAll{$track$ \textbf{in} $lastPlayedTracks$}
			\ForAll{$artist$ \textbf{in} $track.Artists$}
				\ForAll{$relatedArtist$ \textbf{in} $artist.relatedArtists$}
					\If{$relatedArtists.contains(relatedArtist)$}
						\State{$relatedArtist.count += 1$}
					\Else{}
						\State{$relatedArtists.add(relatedArtist)$}
						\State{$relatedArtist.count = 1$}
					\EndIf{}
				\EndFor{}
			\EndFor{}
		\EndFor{}
		\State{$mostRelated = null$}
		\ForAll{$artist$ \textbf{in} $relatedArtists$}
			\If{$mostRelated == null$ || $artist.count > mostRelated.count$}
				\State{$mostRelated = artist$}
			\EndIf{}
		\EndFor{}
		\ForAll{$track$ \textbf{in} $mostRelated.topTracks$}
			\If{$!lastPlayedTracks.contains(track)$}
			\State{}\Return{$track$}
			\EndIf{}
		\EndFor{}
	\EndFunction{}
\end{algorithmic}
\end{algorithm}
