\section{Client Application}
\label{ssub:client_application}

When designing a client application, multiple aspects regarding the usage of the application have to be considered. These include where the application will be used and by whom. The following sections list the choices made in the development of the client application.

\subsection{Native Application or Web Application}
\label{par:native_application_or_web_application}

As described in \cref{sub:pact_context} the client application is used in bars and pubs. Therefore the application has to run on a mobile platform. As the mobile platform is in rapid growth, new possibilities appear and disappear quickly. However smartphones have become common property, as described in \cref{sub:smartphone_usage}. Applications on smartphones can either be native applications that are installed on the device or be web applications that are accessed through the device's browser. The key differences between the two types of applications can be summarised by; native applications often feel more responsive and look more integrated in the smartphone environment. On the downside  getting native applications to work on multiple platforms is usually very time consuming. Web applications on the other hand work on most, if not all, devices~\cite{charland2011mobile}. Web applications, although easier to create and access, are typically less responsive and the possibilities for accessing information about the device it is being used on is severely limited.

As the responsiveness of the application is important for a pleasant user experience, as concluded in \cref{pact}, it is chosen to develop the client application as a native application. By making it native to the platform, the look and feel of the application is more familiar to the user because user interface elements share the same traits as the rest of the well-known platform.