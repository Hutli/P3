\section{Client Application}
\label{ssub:client_application}

When designing a client application, multiple aspects regarding the usage of the application has to be considered. These include where the application will be used and by whom. The following paragraphs list the choices made in the development of the client application.

\subsection{Native Application or Web Application}
\label{par:native_application_or_web_application}

As described in \cref{sub:pact_context} the client application is used in bars and pubs. Therefore the application has to run on a mobile platform. As the mobile platform is in rapid growth, new possibilities appear and disappear quickly. However smartphones have become common property, with 73\% of households in Denmark owning a smartphone \cite{smartphone2014}. Applications on smartphones can either be native applications that are installed on the device or be web applications that are accessed through the device's browser. The key differences between the two types of applications can be summarised by; native applications often feel more responsive and look more integrated in the smartphone environment. On the downside native applications have to be installed on the device, taking time to install and filling up the smartphone's internal storage, and getting the application to work on multiple platforms is usually very time consuming. Web applications on the other hand are ready in an instant~\cite{charland2011mobile} and work on most, if not all, devices. Web applications, although easier to create and access, are typically less responsive and the possibilities for accessing information about the device it is being used on is severely limited.

As the responsiveness of the application is important for a pleasant user experience (as was concluded in \cref{pact}), it is chosen to develop the client application as a native application. By making it native to the platform, the look and feel of the application is more familiar to the user because user interface elements share the same traits as the rest of the well-known platform.

\subsection{User Authorization}
\label{par:user_authorization}

In order to differentiate users of the client application, some mechanism of authorization is needed in the client application. Possible mechanisms are: logging in with already established login services like Google login and Facebook login, logging in with a self made login system or basing the authorization of a device id e.g. IMEI number or a phone number.

The good thing about reusing login services like Facebook login or Google login which are well known is that users do not need to create a new account just for using the client application. Another benefit is that security is taken care of externally i.e. secure password storage is not a concern in the development of this application. Some users might have a stigma against these services or fear of their privacy by using the services e.g. that the client application would post on their Facebook wall.

The only benefit of creating a self made login system is that there is more control about the user's accounts and the users do not have to have fear the application posting on their Facebook wall. However the benefits from external login systems are lost i.e. password storage would have to be implemented in a secure way and kept up to par with the latest exploits. Users would also have to create an account solely for the application.

By using a device constant like the IMEI number, the authorization is automatic so the user does not need to do any login. This greatly simplifies the authorization both for the user and for the developer. A limitation of this approach is that all user data associated with the device constant is lost if the user buys a new device, and thereby uses a new device constant.

Based on the aforementioned differences of the three approaches to user authorization, the sensible choice is reusing existing login services. This is not as simple as using a device constant, but it is more reliable as an authorization mechanism. The concern regarding fear of the client application posting on e.g. Facebook walls can be eliminated by not requesting permission to post on the users walls.
