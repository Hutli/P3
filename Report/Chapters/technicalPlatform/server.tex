\section{Backend server}
\label{techPlat:backendServer}

A backend server is required to communicate with clients. This section describes the technical decisions, and complementary argumentation regarding:
\begin{itemize}
  \item How to communicate between client and server
  \item How to output audio in the system
  \item Which component of the system should be responsible for searching
\end{itemize}

\subsection{Communication with Client}
The following events can occur on clients that will have to be
transmitted to the server for further processing.

\begin{itemize}
\item Check-in
\item Searching
\item Voting
\item Influence volume
\item Check-out
\end{itemize}

The following resources are available on the server, accessible from
clients.

\begin{itemize}
\item Playlist
\item History of last played tracks
\item Now playing track
\item Current volume level
\end{itemize}

These events and resources are all available for the client to access from the server
via HTTP endpoints. See \cref{imp:backendServer} for further elaboration.

\subsection{Playback of Music}
When receiving audio data streams from Spotify, some way of outputting the audio is needed. The data stream is of type Pulse Code Modulation (PCM). In searching for a way to playback PCM data it was found that NAudio\footnote{\url{https://naudio.codeplex.com/}} was the most sensible choice in terms of extensibility.

\subsection{Searching}
As the user has to be able search for music in order to request tracks, different ways of giving this ability to the user was possible. In deciding how to search for tracks, two possibilities were immediately apparent:

\begin{itemize}
  \item The client sends a search query to the backend server. The
    server fetches search results matching the query from Spotify's servers. These results are restricted against a chosen restriction. Lastly the restricted results are sent to the client.
  \item The backend server sends the restriction to the client and the client searches and applies the sent restriction.
\end{itemize}

In favour of letting the backend server handle the search would be
that less computations would have to be done on the client, minimising
battery usage on the client. This benefit fits the requirements extracted from the PACT analysis \cref{sec:pact_summary}.

Letting the clients handle the search meant better scalability,
because the backend does not have to handle the individual queries of
each client, but only distribute a restriction and receive vote
inquiries. The restriction would not have to be distributed all the
time, but only  when the client checks in at the bar and when the
restrictions are updated on the server. This way the search would be
quicker by minimizing the number of systems involved in the sending and recieving a search query, given that
the client's CPU can restrict results faster than receiving pre-restricted results from the backend server.

It was chosen to let the backend server handle the search, in which it would minimize the number of necessary computations needed to be done on the client, to compute a restricted search, instead leaving the server with the responsibility of doing the restrictions. This also avoids the issue of having to make sure that the clients always have an updated list of restrictions.
