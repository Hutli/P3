\subsection{Backend server}

A backend server is required to receive requests from the client, this section holds the technical decisions, and complementary argumentation regarding: 
\begin{itemize}
  \item How to communicate between client and server
  \item How to output audio in the system
  \item What side of the system should be responsible for searching
\end{itemize}

\paragraph{Searching}
As the user has to be able search for music in order to request tracks, different ways of giving this ability to the user was possible. In deciding how to search for tracks, two possibilities were immediately apparent: 
\begin{itemize}
  \item The client sends a search query on the backend, the backend applies the filter and return the results to the client 
  \item The backend sends the filter to the client and the client searches and applies the filter
\end{itemize}
Letting the clients handle the search meant better scalability, because the backend does not have to handle the individual queries of each client, but only distribute a filter and receive vote enquiries. The filter would not have to be distributed all the time, only once when the client checks in at the bar and when the filter is updated, this way the search would be quicker by minimizing the systems involved in the system, given that the client's CPU can filter results faster than receiving result from an external source. 

In favour of letting the backend handle the search would be that less computations would have to be done on the client, minimizing battery usage and only relevant search results(already filtered on the backend) would be downloaded, which could potentially save network bandwidth for the user. 