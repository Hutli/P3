\section{Backend server}
\label{techPlat:backendServer}

A backend server is required to receive requests from the client, this section holds the technical decisions, and complementary argumentation regarding:
\begin{itemize}
  \item How to communicate between client and server
  \item How to output audio in the system
  \item What side of the system should be responsible for searching
\end{itemize}

\subsection{Communication with Client}
The following events can occur on the client that will have to be
transmitted to the server for further processing.

\begin{itemize}
\item Check-in
\item Searching
\item Voting
\item Check-out
\end{itemize}

These events are all available for the client to access from the server
via HTTP endpoints. See \cref{imp:backendServer} for further elaboration.

\subsection{Playback of Music}
When receiving the data stream from Spotify, some way of outputting the audio is needed. The data stream is of type Pulse Code Modulation (PCM). In searching for a way to playback PCM data it was found that Naudio\footnote{\url{https://naudio.codeplex.com/}} was most sensible choice in terms of extensibility.

\subsection{Searching}
As the user has to be able search for music in order to request tracks, different ways of giving this ability to the user was possible. In deciding how to search for tracks, two possibilities were immediately apparent:
\begin{itemize}
  \item The client sends a search query to the backend. The backend fetches search results matching the query from Spotify's servers. These results are filtered against a chosen filter. Lastly the filtered results are sent to the client.
  \item The backend sends the filter to the client and the client searches and applies the sent filter.
\end{itemize}

In favour of letting the backend handle the search would be that less computations would have to be done on the client, minimising battery usage and only relevant search results, already filtered on the backend, would be downloaded, which could potentially save network bandwidth for the client.

Letting the clients handle the search meant better scalability, because the backend does not have to handle the individual queries of each client, but only distribute a filter and receive vote inquiries. The filter would not have to be distributed all the time, only once when the client checks in at the bar and when the filter is updated, this way the search would be quicker by minimizing the systems involved in the system, given that the client's CPU can filter results faster than receiving result from an external source.

For this project, the
