\section{Usability}
\label{sub:usability}

Usability can be defined as: \enquote{when a product or service is truly usable, the user can do what he or she wants to do the way he or she expect to be able to do it, without hindrance, hesitation, or questions}~\cite{RubinChisnellSpool08}.

There are different ways of evaluating the usability of a user interface. The two being looked at in this report are empirically and heuristically testing. The goal of a usability test is to come up with a list of usability problems. A usability problem is a deficiency in a piece of software that makes it less usable. An example of an usability problem is when a simple task like deleting an email takes an inappropriate amount of effort.

\subsection{Usability Testing}
Usability Testing is an empirical technique used to evaluate a product by testing it with users in focus. A usability test is usually performed with a representative user performing a series of tasks. A observant will watch the participant completing the tasks, without interrupting the participant in doing the tasks. The purpose of the test is to determine usability problems with the product~\cite{RubinChisnellSpool08}. In this report the term \emph{usability testing} is used to refer to the process of testing with users, and not as a general term for testing usability. 

Usability testing is intended to be realistic and a potential scenarios, that can that can happened under use of the product when it reaches the end user.

After a usability test is conducted the results must be analysed and documented. This usually involves spending a lot of time doing video data analysis. In \cite{kjeldskov2004instant} a new method called Instant Data Analysis (IDA) is presented. This method seeks to reduce the resources required to conduct analysis. In IDA the procedure is as follows: After the test a 1 hr brainstorm session between the data logger and test monitor. There will also be a facilitator that manages the brainstorm and write usability problems on a whiteboard. After the brainstorm the facilitator spends 1-1½ hours writing the usability problems into a ranked list. \cite{kjeldskov2004instant} found that this method reduced the time to do analysis to 10\% to that of traditional video analysis. While still finding 85\% of the critical problems. This makes it a easy way of finding problems but can lack in other areas.

\frnote{noget med IDA} 

\subsection{Usability in context}
\label{sub:usability_in_context}

Due to the fact that it often more difficult to use a phone while influence of alcohol, the product has to be user friendly, even if the user has been drinking.

Usability testing in this project will be used as a tool to test and ensure quality of the user interface. The test will be performed with a some typical user that was determined in \cref{userInterviews}, which are representative end users of the product.

The test will be recorded, and actions performed on the application logged for later analysis. By performing the task in this way, the observant do not have to be in the same room as the person performing the task. Further more the task will written simplified and with as little details as possible, to force the participant to explore the application by themself. The result of this can be used as an indication of how intuitively the product is.
\frnote{Hvorfor er det en god måde at gøre det på? Har vi en kilder der siger noget om the?}

\subsection{Heuristic Evaluation}
Heuristic evaluation is a method to inspect usability of computer software. It involves a number of evaluators that are presented with an interface design and asked to comment on it. This is used to come up with a list of usability problems in a user interface design. This information can be used to as part of an iterative design process make the design more usable. Heuristic evaluation is a cheap and intuitive way of doing usability evaluation and it does not require advanced planing~\cite{Nielsen1990}.

Heuristic evaluation is usually done by judging how the software meets certain predetermined heuristic. \cite{Nielsen1990} introduced a set of 10 heuristics that was then later updated in~\cite{Nielsen1994}, that are as follows:

\begin{enumerate}
  \item Visibility of system status
  \item Match between system and the real world
  \item User control and freedom
  \item Consistency and standards
  \item Error prevention
  \item Recognition rather than recall
  \item Flexibility and efficiency of use
  \item Aesthetic and minimalist design
  \item Help users recognize, diagnose, and recover from errors
  \item Help and documentation
\end{enumerate}

The method is not good for finding all problems; an individual person can only find between 20\% and 51\% of the problems~\cite{Nielsen1990}. Therefore it is recommended that multiple people conduct the evaluation independently of another. The results can then be compiled together. \cite{Nielsen1990} recommends that 3 to 5 people do independent evaluation to get good coverage.

A disadvantage of heuristic evaluation is that it sometimes identifies usability problems without providing direct suggestions for how to solve them~\cite{Nielsen1990}. The method is biased by the current mindset of the evaluators and normally does not generate breakthroughs in the evaluated design.

\subsection{Cool title}
The first usability evaluations that was done was heuristic evaluation. This was done because it can be used early in the development process and it is a good fit for the iterative development model that was used in the project.
