\section{Usability}
\label{sub:usability}

Usability can be defined as: \enquote{when a product or service is truly usable, the user can do what he or she wants to do the way he or she expect to be able to do it, without hindrance, hesitation, or questions}~\cite{RubinChisnellSpool08}.

There are different ways of evaluating the usability of a user interface. The two being looked at in this report are empirically and heuristically testing. The goal of a usability test is to come up with a list of usability problems. A usability problem is a deficiency in a piece of software that makes it less usable. An example of a usability problem is when a simple task like deleting an email takes an inappropriate amount of effort.

\subsection{Usability Testing}
\label{sub:usabilityTesting}
Usability testing is an empirical technique used to evaluate a product by testing it with users in focus. The purpose of the test is to determine usability problems with the product~\cite{RubinChisnellSpool08}. In this report the term \emph{usability testing} is used to refer to the process of testing with users, and not as a general term for testing usability. The test is intended to be realistic and simulate potential scenarios that can happen while using the product. A usability test is usually performed with a representative user performing a series of tasks. A test monitor will watch the participant completing the tasks without interrupting the participant in doing the tasks. While the participant is solving the task, a data logger is taking notes about how the participant is solving the task given and the problems faced doing so.

After a usability test is conducted the results must be analysed and documented. In most usability testing analysis methods, this involves spending a lot of time doing video data analysis. In~\cite{kjeldskov2004instant} a new method called \enquote{Instant Data Analysis} (IDA) is presented. This method seeks to reduce the resources required to conduct the analysis. In IDA the procedure is as follows. After the test, a one hour brainstorm session between the data logger and test monitor has to occur. A facilitator  manages the brainstorm and writes usability problems on a whiteboard. After the brainstorm the facilitator spends 1-1.5 hours writing usability problems into a ranked list.~\cite{kjeldskov2004instant} found that this method reduced the time to do analysis to 10\% of the traditional video analysis. While still finding 85\% of the critical problems.

\subsection{Heuristic Evaluation}
Heuristic evaluation is a method to inspect usability of computer
software. It involves a number of evaluators that are presented with
an interface design and asked to comment on it. This is used to come
up with a list of usability problems in a user interface design. This
information can be used as part of an iterative design process to make the design more usable. Heuristic evaluation is a cheap and intuitive way of doing usability evaluation and it does not require advanced planning~\cite{Nielsen1990}.

Heuristic evaluation is usually done by judging how the software meets certain predetermined heuristics.~\cite{Nielsen1990} introduced a set of ten heuristics that was then later updated in~\cite{Nielsen1994}, that are as follows:

\begin{enumerate}
  \item Visibility of system status
  \item Match between system and the real world
  \item User control and freedom
  \item Consistency and standards
  \item Error prevention
  \item Recognition rather than recall
  \item Flexibility and efficiency of use
  \item Aesthetic and minimalist design
  \item Help users recognise, diagnose, and recover from errors
  \item Help and documentation
\end{enumerate}

The method is not good for finding all problems; an individual person can only find between 20\% and 51\% of all problems~\cite{Nielsen1990}. Therefore it is recommended that multiple people conduct the evaluation independently of another. The results can then be compiled together.~\cite{Nielsen1990} recommends that 3 to 5 people do independent evaluation to get good coverage.

A disadvantage of heuristic evaluation is that it sometimes identifies usability problems without providing direct suggestions for how to solve them~\cite{Nielsen1990}. The method is biased by the current mindset of the evaluators and normally does not generate breakthroughs in the evaluated design.

\section{Usability Test in Laboratory}
Usability testing in this project will be used as a tool to test and
ensure quality of the user interface. The tests will be performed on
typical users that were determined in \cref{userInterviews}, who are
representative end users of the product. The outcome of a usability
test is a list of usability problems. It was chosen to conduct the
test in a laboratory since it is easy to create a controllable
environment and record video. Because of the limited time, it was
chosen to do IDA as described in \cref{sub:usabilityTesting}. This
might not find all usability problems, but this was acceptable since
only a rough overview of the usability of the product was needed.

\subsection{Procedure}
The test was done on a smartphone of the model OnePlus One. This was
the phone that was available at hand and it is equipped with a quick
processor. That way performance will not be a problem in the
test. After all the test is testing the software and not the
hardware. The members of the group were assigned different roles; two
persons were assigned as test monitors and alternately were also the
leader in the subject room. Another person was datalogger and was in
the control room taking notes. A fourth person was
smoothtalker/driver, so he was responsible for entertaining the test
subjects when they were not being tested and also driving the subjects
that needed help arriving at the testing location. The fifth person
was the video operator and was responsible for recording and mixing
the video feeds from the subject room. The last person was assigned
the role of social environment simulator and was therefore responsible
for simulating user users actively using the test application.

At the start of a test, an introduction was read out loud to the test
subject. The introduction can be found in
\cref{usabilityTestIntro}. The subject would then answer a
questionnaire which can be found in \cref{tab:participants}. To
simulate a bar the subject was offered drinks and crisps. The test
proceeded with the test subjects reading a premade task and then
trying to accomplish that task. All tasks can be seen in \cref{usability_tasks}. According to~\cite{RubinChisnellSpool08}, the task description has to be detailed enough, so that no excessive information about the test program is exposed. The result of this can be used as an indication of how intuitively the product is.
\frnote{billede af testrum}

\subsection{Participants}
It was decided to test with users, because it would give an indication
of how real end-users use the system. Since the system is to be used
in a bar and is targeted for students, students were chosen as test
subjects. The subjects were recruited from the user interviews
described in \cref{userInterviews}. \Cref{tab:participants} presents
information gathered from the before mentioned questionnaire.

\begin{table}[h]
\begin{tabular}{|l|l|l|l|l|l|}
\hline
\textbf{\#} & \textbf{Sex} & \textbf{Age} & \textbf{Field of study} & \textbf{Smartphone experience} & \textbf{Primary phone} \\ \hline
1                   & F               & 20           & Techno-Anthropology       & Experienced                    & iOS                    \\ \hline
2                   & F               & 19           & Mathematics             & Intermediate                   & Android, iOS           \\ \hline
3                   & F               & 20           & SIV Spanish             & Experienced                    & iOS                    \\ \hline
4                   & F               & 22           & Electronics and IT      & Intermediate                   & Android                \\ \hline
5                   & M               & 20           & Electronics and IT      & Experienced                    & Android                \\ \hline
\end{tabular}
\caption{Participants}\label{tab:participants}
\end{table}

\subsection{Data Collection}
The screen of the phone was recorded with software on the phone. The
video feeds from the two cameras in the test room were recorded to a
DVD. The data logger was taking notes while the test subjects were
performing their tasks.

\subsection{Data Analysis}
The IDA brainstorming and data analysis session was conducted on the next
day, and resulted in the following list of problems. The ranking of
the problem, a problem description and which test persons suffered
from the problem is noted.

\subsubsection{Serious}
\textbf{Problem \#1}\\
    Vote confusion - Does not know how votes work, what the person has
    voted for and when the person has voted. Test persons \#1, \#2,
    \#3, \#4 and \#5

\noindent\textbf{Problem \#2}\\
    Confusion about multiple copies of the same song on the playlist
    and now playing. Test persons \#2, \#4 and \#5

\noindent\textbf{Problem \#3}\\
    The user is confused on how well tracks are doing on the
    playlist. Test persons \#1, \#2 and \#5.

\subsubsection{Cosmetic}
\textbf{Problem \#4}\\
    Missing confirmation when adding tracks on search. Test persons
    \#1, \#3, \#4 and \#5

\noindent\textbf{Problem \#5}\\
    Check-in confusion - Does not know whether the person is checked
    in or not. Test persons \#1, \#2 and \#3.

\noindent\textbf{Problem \#6}\\
    Did not know in which direction to swipe on the start page. Test
    persons \#2, \#4 and \#5.

\noindent\textbf{Problem \#7}\\
    Search confusion: Does not know whether and when a search is
    happening. Test persons \#1 and \#5.

\noindent\textbf{Problem \#8}\\
    Loading indicator confuses user. They thought it was their actions
    that influenced the loading indicator. Test person \#2.

\noindent\textbf{Problem \#9}\\
    Expected “clear search button” to remove results and bring user
    back to the playlist. Test person \#2.

\noindent\textbf{Problem \#10}\\
    The IP-address in the venue page is confusing. They thought it was
    the number of guests at the venue. Test persons \#1 and \#2.

\noindent\textbf{Problem \#11}\\
    Confusion about which votes were made by users and which were
    votes made by the bar. Test person \#2.

\noindent\textbf{Problem \#12}\\
    Did not know what tracks in a search already existed on the
    playlist. Test person \#2.

\noindent\textbf{Problem \#13}\\
    Was confused why the search did not clear when the user made a new
    search. Test person \#2.

\noindent\textbf{Problem \#14}\\
    Excepted ability to get more info when interacting with all
    tracks. Test person \#5.

\subsubsection{Usability Test Summary}

As no critical problems were found in the use of the system, the
usability of the system is concluded to be good, but certainly not
perfect. The problems found can be discussed and eventually
solved, leading to a more usable system.
