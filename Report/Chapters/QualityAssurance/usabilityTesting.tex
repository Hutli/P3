\section{Usability}
\label{sub:usability}

Usability can be defined as: \enquote{when a product or service is truly usable, the user can do what he or she wants to do the way he or she expect to be able to do it, without hindrance, hesitation, or questions}~\cite{RubinChisnellSpool08}.

There are different ways of evaluating the usability of a user interface. The two being looked at in this report are empirically and heuristically testing. The goal of a usability test is to come up with a list of usability problems. A usability problem is a deficiency in a piece of software that makes it less usable. An example of a usability problem is when a simple task like deleting an email takes an inappropriate amount of effort.

\subsection{Usability Testing}
Usability testing is an empirical technique used to evaluate a product by testing it with users in focus. The purpose of the test is to determine usability problems with the product~\cite{RubinChisnellSpool08}. In this report the term \emph{usability testing} is used to refer to the process of testing with users, and not as a general term for testing usability. The test is intended to be realistic and simulate potential scenarios that can happen while using the product. A usability test is usually performed with a representative user performing a series of tasks. A test monitor will watch the participant completing the tasks without interrupting the participant in doing the tasks. While the participant is solving the task, a data logger is taking notes about how the participant is solving the task given and the problems faced doing so.

After a usability test is conducted the results must be analysed and documented. In most usability testing analysis methods, this involves spending a lot of time doing video data analysis. In~\cite{kjeldskov2004instant} a new method called \enquote{Instant Data Analysis} (IDA) is presented. This method seeks to reduce the resources required to conduct the analysis. In IDA the procedure is as follows. After the test, a one hour brainstorm session between the data logger and test monitor has to occur. A facilitator  manages the brainstorm and writes usability problems on a whiteboard. After the brainstorm the facilitator spends 1-1.5 hours writing usability problems into a ranked list.~\cite{kjeldskov2004instant} found that this method reduced the time to do analysis to 10\% to that of traditional video analysis. While still finding 85\% of the critical problems.

\subsection{Usability in Context}
\label{sub:usability_in_context}

Due to the fact that it is often more difficult to use a phone under influence of alcohol, the product has to be user friendly.

Usability testing in this project will be used as a tool to test and ensure quality of the user interface. The tests will be performed on typical users that were determined in \cref{userInterviews}, which are representative end users of the product.

The test will be recorded, and actions performed on the application logged for later analysis. As every action is logged on the application, the observant does not have to be in the same room as the person performing the task. According to \cite{RubinChisnellSpool08}, the task description has to detail just enough of the task, so that no excessive information about the test program is exposed. The result of this can be used as an indication of how intuitively the product is.

\subsection{Heuristic Evaluation}
Heuristic evaluation is a method to inspect usability of computer software. It involves a number of evaluators that are presented with an interface design and asked to comment on it. This is used to come up with a list of usability problems in a user interface design. This information can be used to as part of an iterative design process make the design more usable. Heuristic evaluation is a cheap and intuitive way of doing usability evaluation and it does not require advanced planning~\cite{Nielsen1990}.

Heuristic evaluation is usually done by judging how the software meets certain predetermined heuristics.~\cite{Nielsen1990} introduced a set of ten heuristics that was then later updated in~\cite{Nielsen1994}, that are as follows:

\begin{enumerate}
  \item Visibility of system status
  \item Match between system and the real world
  \item User control and freedom
  \item Consistency and standards
  \item Error prevention
  \item Recognition rather than recall
  \item Flexibility and efficiency of use
  \item Aesthetic and minimalist design
  \item Help users recognise, diagnose, and recover from errors
  \item Help and documentation
\end{enumerate}

The method is not good for finding all problems; an individual person can only find between 20\% and 51\% of all problems~\cite{Nielsen1990}. Therefore it is recommended that multiple people conduct the evaluation independently of another. The results can then be compiled together.~\cite{Nielsen1990} recommends that 3 to 5 people do independent evaluation to get good coverage.

A disadvantage of heuristic evaluation is that it sometimes identifies usability problems without providing direct suggestions for how to solve them~\cite{Nielsen1990}. The method is biased by the current mindset of the evaluators and normally does not generate breakthroughs in the evaluated design.

\subsection{Cool title}
The first usability evaluations that was done was heuristic evaluation. This was done because it can be used early in the development process and it is a good fit for the iterative development model that was used in the project.

\sinote{Todo egen usability test}
