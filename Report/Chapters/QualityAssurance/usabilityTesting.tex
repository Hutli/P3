\subsection{Usability Testing}
\label{sub:usability_testing}

Usability Testing is a technique used to evaluate a product by testing it with users in focus. A usability test is usually preformed with a representative user preforming a series of tasks. A observant will watch the participant completing the tasks, without interrupting the participant in doing the tasks. The purpose of the test is to determine usability problems with the product. 
Usability testing is intended to be realistic and a potential scenarios, that can that can happened under use of the product when it reaches the end user. 

\subsection{Usability in context}
\label{sub:usability_in_context}

Due to the fact that it often more difficult to use a phone while influence of alcohol, the product has to be user friendly, even if the user has been drinking.  
Usability testing will in this project be used as a tool to test and ensure quality of the user interface. The test will be preformed with a some typical user that was determined in \cref{userInterviews}, which are representative end users of the product. 


The test will be recorded, and actions preformed on the application logged for later analysis. By preforming the task in this way, the observant do not have to be in the same room as the person preforming the task. Further more the task will written simplified and with as less details as possible, to force the participant to explore the application by him self. The result of this can be used as and indication of how intuitively the product is.

