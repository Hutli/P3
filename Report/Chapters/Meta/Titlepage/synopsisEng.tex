%!TEX root = ../Master.tex
% \begin{center}
% This project focuses on the interaction between the guests and
% administrators at a venue, often a bar or a pub, with focus on music.
% In interviews, it was found that this interaction is currently not a
% problem, since requesting and other music-related interaction is not a
% part of the experience, guests have when visiting venues. This project
% therefore focuses on analysing the context, environment and
% experience, the guests currently have at venues. It aims to find a way
% to extend this experience to also cover music, and integrate this into
% the experience and social environment. This will be done by shifting
% some of the control of music from the administrators to the guests,
% and enable guests to socially participate in the selection process,
% while still maintaining the interests of the venue. It is important to
% keep in mind, throughout the project, that the product should not
% limit the current experience the users have, by for instance
% distracting them.
% \end{center}

\begin{center}
The purpose of this project is to create a common platform for guests at a venue to colaborally influence the music that is playing. The different user groups’ interests were researched and analysed, using interviews, to find various requirements.
Concepts for the system were derived from the aforementioned interviews as well as similar systems.

The conception of the system is based on the concepts of object oriented analysis and design.

The implementation, a software system, consisting of a server and several clients, is created utilising the features of the object oriented paradigm.
The quality of the solution was assured through usability tests,
analysed using Instant Data Analysis. Here no critical issues were
found, so the software system is generally useable. Serious and
cosmetic issues were found and corrected or otherwise noted.
\end{center}
