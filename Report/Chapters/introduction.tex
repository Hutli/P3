\chapter{Introduction}\label{introduction} John was having a good time
at his favourite bar. He was enjoying a cool beverage and was in the
mood for his favourite track, \enquote{Carry On My Wayward Son}. John
went to the bartender to give his request, but when he got to the
counter, there was a long queue, which he had to wait for. After John
had waited for what felt like an eternity, the bartender finally walked
towards him. John could give his request at last. He greeted the
bartender with a passion and enthusiastically spoke his request. The
bartender politely turned his request down, as it did not fit into the
general theme the bar was going for.

There are several problems, which could have been avoided, with what
happened to John.
\begin{enumerate}
	\item He had to stand and wait in line at the bar
	\item He had to go to the bar, potentially leaving his seat and/or
company
	\item He had to bother the bartender with something which is
not the bartender's primary function
	\item His time was wasted, as his request was denied
\end{enumerate}

When people go to a bar or similar venue, they are looking for an
experience. This project seeks to analyse the context of this
experience, and eliminate all four of the above mentioned problems,
thereby enhancing the experience through music and possibly social
activity and interaction.\\

This project uses the ideas described in \cite{sorensen2012} as the
base inspiration. \cite{sorensen2012} presents the idea of a multi-device system in
which every device can influence music at a particular location. The
focus is on the interaction between every device, and trying to form a
unified system of all the devices.

Having these ideas in mind at the start of the
project, they were refined and extended through interviews with
administrators of venues and potential users. The initial idea from
\cite{sorensen2012} evolved during the analysis and design of the
project into not only creating an experience for the users, but also
pursuing the interests of venues.

Ultimately the software described in the project is able to manage
the music playing at a venue or for a large crowd. Every user of the
software can influence the music being played through a cross platform
mobile application. While this allows for great freedom for users,
administrators at venues are still able to protect their musical
preferences.
