\chapter{Introduction}\label{introduction}
John was having a good time at his favourite bar. He was enjoying a cool beverage and was in the mood for his favourite track, \enquote{Carry On My Wayward Son}. John went to the bartender to give his request, but when he got to the counter, there was a long queue, which he had to wait for. After John had waited for what felt like an eternity, the bartender finally walked towards him. John could give his request at last. He greeted the bartender with a passion and enthusiastically spoke his request. The bartender politely turned his request down, as it did not fit into the general theme the bar was going for.

There are several problems, which could have been avoided, with what
happened to John.

\begin{enumerate}
	\item He had to stand and wait in line at the bar
	\item He had to go to the bar, potentially leaving his seat and/or
company
	\item He had to bother the bartender with something which is not the bartender's primary function
	\item His time was wasted, as his request was denied
\end{enumerate}

When people go to a bar or similar venue, they are looking for an experience. This project seeks to analyse the context of this experience, and eliminate all four of the above mentioned problems, thereby enhancing the experience through music and possibly social activity and interaction.\\

This project uses \citetitle{sorensen2012}~\cite{sorensen2012} as the base inspiration.
Some ideas, that spawned from \citetitle{sorensen2012}, are based on multiple devices coming together to form a unified system, influencing the music at a particular location. This means creating a system, which functionality is not based on any single device, but rather the communication between the different devices in the system.
This is done through the devices’ internet connection, where each device communicates to and receives data from the system. These devices can be smartphones, from which a user can interact with the system, but can also be situated displays, where multiple users can perceive the current state of the system. The center of the whole system is a server running on a PC, containing all the data about the system and is what all other devices communicate with. Based on static documented in the report most people in Denmark, owns a smartphone. This makes the interaction between the user and the system possible through a smartphone based solution.  

The system should not be the center of attention, so as to not lessen the social experience, which exists in the context of bar-like venues.

These ideas were refined and extended through interviews with administrators of venues and potential users. The initial idea from \citetitle{sorensen2012} evolved during the analysis and design of the project into not only creating an experience for the users, but also pursuing the interests of venues.

Ultimately the system described in the project is able to manage the music playing at a venue or for a large crowd. Every user of the system can vote for tracks, through a cross platform mobile application, which will influence the music that is playing at the venue. While this allows the users great freedom to influence the music in the direction they want, administrators at venues are still able to protect their musical preferences.
