As stated in \cref{par:music_catalog_libspotify} the only way to stream tracks off Spotify is using the C library libspotify. Being written as a C library, one has to take care of managing memory allocation correctly in order to avoid runtime exceptions while using libspotify. 

To make it harder to cause these runtime errors and to ease interoperability in the C\# environment, a C\# library can be written that abstracts the error prone elements of C programs away. The C\# library \enquote{SpotifyDotNet} developed for use in the software described in this paper, does just that.

\subsubsection{Abstracting Low-level C Concepts Away}
\label{ssub:abstracting_low_level_c_concepts_away}

The goal of the SpofityDotNet library is to minimize runtime exceptions caused by memory related errors. To do this, the error prone unmanaged C code has to be accessed as secure managed C\# code.

SpotifyDotNet uses existing bindings, hosted on \url{http://libspotifydotnet.codeplex.com/} \sinote{Hvad med licens?}, to interface with libspotify from C\#. As an example let's look at the login function from libspotify. See \cref{fig:loginC}. The corresponding binding can be seen in \cref{fig:loginCsharp}. As can be seen, a C pointer maps directly to an IntPtr struct in C\#. \enquote{const char*} becomes a common string object in C\#. A C bool is simply a C\# bool.

\begin{lstlisting}[caption = {Libspotify login function prototype - C}, label = {fig:loginC}]
sp_error sp_session_login ( sp_session *session,
                            const char *username,
                            const char *password,
                            bool remember_me,
                            const char *blob
                          )
\end{lstlisting}

\begin{lstlisting}[caption = {Login method using a external implementation from libspotify.dll - C\#}, label={fig:loginCsharp}]
public static extern sp_error sp_session_login ( IntPtr sessionPtr, 
                                                 string username, 
                                                 string password, 
                                                 bool rememberMe, 
                                                 string blob 
                                               )
\end{lstlisting}

However this is not abstracted enough. To the user of the library, no pointers should be visible. This is solved by encapsulating state and data inside the Spotify class. This class is the entry point to using libspotify in C\#. See \cref{fig:spotifydotnet_class} for a complete class diagrams. The main operation on the class is the Login method. This method is asynchronous so will return execution to the caller right away. When login has occurred, either OnLogInSuccess or OnLogInError callbacks will be called. These are implemented as C\# events, making it possible for multiple subscribers to listen to this callback. The OnLogInSuccess event sends with it a SpotifyLoggedIn object, also seen in \cref{fig:spotifydotnet_class}, which has methods to be used only when correctly logged into Spotify. By only exposing the SpotifyLoggedIn object when correctly logged in, certain errors like searching without being logged in, can be detected at compile time.

\begin{figure}
  \centering
  \includegraphics[width=1.2\linewidth]{spotifydotnetClass.png}
  \caption{Class diagram of SpotifyDotNet library}
  \label{fig:spotifydotnet_class}
\end{figure}

\subsubsection{Making it Thread-safe}
\label{ssub:making_it_thread_safe}

As stated in~\cite{spotifyLibspotifyFAQ}, libspotify is not thread safe. This is solved in SpotifyDotNet by using locks around non thread safe code. This is easily done in C\# using the \enquote{lock} keyword as seen in \cref{lst:lock_keyword}.

\begin{lstlisting}[caption = {Example of using the lock keyword in C\#. \enquote{\_sync} is an object used to store the lock state}, label = {lst:lock_keyword}]
lock(_sync) {
  thisWillRunThreadSafe();
}
\end{lstlisting}

To the user of SpotifyDotNet, no errors related to threads can occur when using the library on multiple threads.
