In the interviews that were conducted with potential users, it was expressed that most people found it distracting and annoying when other people at social venues were using their smartphones too much, instead of interacting with each other at the venue, and that this is a frequent phenomena.

To avoid users getting distracted from their social interaction it was decided that the design of the application should aim to be as minimal and simplistic as possible. To do so, a minimal interaction and information approach was chosen. This means that the user should only get as much information as minimally needed for the current task and the interaction should be kept simple and easy to understand.

Following this approach it was decided that the user will only get one vote per track played. The user can change his or her vote as he or she likes but when the current track ends the information is gathered about what is voted for and the votes are saved permanently. After this, all users' votes are reset and the user can vote again.

To simplify the request system, a request is just a vote on a track that is not on the playlist meaning that when a track, that is not on the playlist, gets a vote it is merely added to the playlist. If a track has 0 votes it is removed from the playlist. This means that the processes of voting and requesting tracks are identical, the only difference being that tracks that have not been requested do not appear on the playlist.