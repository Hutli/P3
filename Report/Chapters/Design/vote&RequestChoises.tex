In the interviews with users it was expressed that most people found it distracting and annoying when other people at social venues where using their smart phones too much instead of interacting with each other at the venue. Furthermore they thought that for some people it could take too much of their attention and found it troubling that this happens too much.

To avoid users getting distracted from their social interaction it was decided that the design of the project should aim to minimize time spent on the app. To do so a minimal interaction and information approach was chosen. This means that the design should always focus on the user should only get as much information as maximum needed for the current task and the interaction should be kept simple and easy to understand.

Accordingly to this approach it was decided that the user will only get one vote per song played. The user can change his og her vote as he or she likes but when the song ends the playlist gathers information about what is voted for and the votes are saved permanently. After this all users temporary votes are reset and the user can vote again.

To simplify the request system, a request is just a vote on a song that is not on the playlist meaning that when a song that is not on the playlist gets a vote it is added. The same goes if a song have 0 votes it is removed from the playlist again. This means that the user are able to vote for a song on the playlist or search for a song and vote for that, resulting in the need for users to understand the concept of request is removed.