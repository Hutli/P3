The project was developed in collaboration with a bar called \enquote{Fabrikken} in the centre of Aalborg. To ensure that it was possible to get feedback from the bar and that the bar should get as little new hardware as possible, in order to implement the system, a minimal hardware approach was chosen.
From this approach it was derived that the input to the user should be decentralised, meaning that each user should be able to see the result of all users' interaction locally on his or her device.
Since the initial problem resided in the administrator being distracted from their main purpose by requests, it was not possible to centralise the interaction on the administrator to solve the problem. Since nearly all venues currently only have a computer playing music, which is only available to the administrator, a centralised device would conflict with the minimum hardware approach. Because of this, it was chosen to decentralise the output from the user as well.

Since both input and output should be decentralised, it is possible to have both in a single device.
To reach the maximum amount of possible users, while still maintaining the decentralised input/output, it was chosen to use smartphones because of the large usage of these among the users\cref{sub:smartphone_usage}. To make the solution possible for as many users as possible the solution will be available across the three main smartphone platforms; Android, iOS and Windows Phone.