The project was developed in collaboration with a bar called \inquote{Fabrikken} in the center of Aalborg. To assure that it was possible to get feedback from the bar and that the bar should get as little new hardware as possible, or non at all, in order to implement the system, a minimal hardware approach where chosen.
From this approach it was derived that the input to the user should be decentralized meaning that each user should be able to see the result of all users interaction local on his or her device. To reach the maximum amount of possible users, while still maintaining the decentralized input, it was chosen to use smartphones because of the large usage of these amoung the users\cref{sub:smartphone_usage}. To make the solution possible for as many users as possible the solution will be awailable across the three main smartphone platforms Android, iOS and Windows Phone.
Since the initial problem resided in the administrator being distracted from their main purpose by requests it was not possible to centralize the interaction on the administrator to solve the problem. Since nearly all venues currently do only have a computer playing music only accessible by the administer and do not have a centralized interaction hardware for users, a centralized device would conflict with the minimum hardware approach since they would have to buy this. It was therefore chosen also to decentralized the interaction (output) from the user to make it possible for all users to interact.

To decentralize both the input and output makes it possible to combine both input and putput in one device. The solution will therefore be a cross-platform smartphone app where the user can vote and request tracks for a playlist that is played and administered by a backend application for the computer currently playing the music at venues.