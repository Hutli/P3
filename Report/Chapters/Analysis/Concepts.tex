When analysing the State of the art it was found that many different concepts was being used in the basis of the design of these different systems. This section focuses on describing each concept and its pros and cons.

\subsection{Voting}
All three systems uses voting in some way. [Systems only using up-votes] only uses up-votes meaning that in these systems people can only vote on songs they like and want to be played sooner. [Systems with down-vote] also adapt the concept of down-voting songs enabling the user to vote song they don't want to hear (or just not yet) down the play-list making them getting played later or not at all.
Voting can also be used, for instance in the case of SecretDJ, to rise the level of another user, explained later, thereby enabeling the user to let his or her songs start higher on the playlist of the specifik venue.

\subsection{Request}
SecretDJ - A user can add 4 songs each day


\subsection{Play-list}


\subsection{Automatic rearrangement of playlist}

\subsection{Editing and restriction of playlist}

\subsection{Levels}

\subsection{Check-in}

\subsection{All time available venue information}

\subsection{Indirect song request and voting}
