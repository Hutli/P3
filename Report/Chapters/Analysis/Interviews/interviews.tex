\label{interviews}
To gain a greater understanding of the problem domain for this project, some interviews were conducted on various bars on the famous street, Jomfru Ane Gade, in Aalborg. The participants include staff from Fabrikken, LA Bar, Newcastle and Dan's Poolhall. The interviews were semi-structured in nature, with questions based in the following categories

\begin{itemize}
  \item Management of the bars music system
  \item Handling requests from the customers
  \item Current and former music systems
  \item Music requirements and the dynamics thereof
\end{itemize}

If some interesting answers to these questions were given, the interviewer would follow up on these answers with some improvised questions. Hence the semi-structured nature of the interviews.

The following subsections describe the similarities and differences between the current music solutions being used at the bars.

\subsection{Similarities}
\label{sub:similarities}
All bars want to be able to control what kind of music is playing. That is, they do not want a slow song playing right after a party hit in the middle of a party night. This is currently managed by prepopulating a playlist from which the music system plays tracks from. Customers can at any time request a track to be played, but no guarantees are given whether the requested song is played or not. When a request is given, the bartender must be the judge of whether or not the requested track is matching the current music theme. All participants' typical customers' age group was around 20-25.


\subsection{Differences}
\label{sub:differences}
Fabrikken and Dan's Poolhall have no DJ employed so their music system is in use throughout their opening hours, while LA Bar and Newcastle have a dedicated DJ playing at night. One participant uses MiB Pro, while the others use Spotify. A remark was made that MiB Pro did not have as large a selection of tracks as could be desired.

\subsection{Specific Remarks}
\label{sub:specific_remarks}

Dan's Poolhall is concerned that streaming music from the internet is risky, because of potential internet failures. A backup solution working offline would be ideal.


% \subsection{Fabrikken}
% \label{sub:fabrikken}

% The interviewee is the owner of the bar. Currently the music system in use is a computer running Spotify with a premium license. Customers have influence regarding the music played, but the songs requested have to be approved by the staff. This is to avoid sudden style in the music, which could ruin the party mood for other customers.

% Most of the customers in the bar are university students. Therefore the owner considers setting up a Wi-Fi network so that during the daytime, when there is a calm mood in the bar, it enables a suitable environment for studying, very much like a café environment.

% \subsection{LA bar}
% \label{sub:la_bar}

% The interviewee is a bartender. The bar uses Spotify at daytime. The bartender does not need to maintain the Spotify music system other than switching songs upon acceptable requests from customers. The playlists being used are prepopulated by the owner of the bar.

% The age group of the customers is 20-30 years, but sometimes married couples also enjoy having a beer in the afternoon.

% At night when the party mood is on the rise, a dedicated DJ manages the music.

% \subsection{Newcastle}
% \label{sub:newcastle}

% The interviewee is a bartender. The music system in use is MiB Pro, which runs on a computer behind the bar desk. Because of this, only the bartenders can control the system. The playlists being played from are prepopulated by the owner of the bar. When asked on problems with the system, it was pointed out that there is a little shortage in the music repertoire. Customers can, like in the other bars, influence the music by requesting songs as long as the songs fits in the overall theme of the bar.

% Similar to LA bar, a DJ is administrating the music every night.

% When asked on an opinion of the features of the system described in this report, the concept was received well, but some issues were raised. The software system would only be used outside of the DJ playtime at night, before the crazy party atmosphere arise.

% \subsection{Dan's Poolhall}
% \label{sub:dan_s_poolhall}

% The interviewee is the bar manager. Spotify is their current music system and is used in the same manner as in the other bars. The staff is making their own playlists, and song requests are possible from the customers. A problem pointed out in the use of Spotify is the need for a stable internet connection, in order for acceptable playback. 

% When confronted with the features of the software system described in this paper, the overall response was positive, but some problems came to mind. The system would have to avoid interuptions in genres and slow songs being played in the middle of a party.

\subsection{Summary}
\label{sub:summary}
From these interviews, some requirements have been gathered based on the participants' responses. These requirements are listed beneath using the MoSCoW method, sorted by importance according to the participants and plausibility.
\subsubsection{Must have}
\begin{itemize}
	\item Ability for the bar to control what music is being played.
	\item Ensured continuity in the tracks that are played
\end{itemize}
\subsubsection{Should have}
\begin{itemize}
	\item No need for bartenders to judge requests from customers
\end{itemize}
\subsubsection{Could have}
\begin{itemize}
	\item No need for a prepopulated playlist
\end{itemize}
\subsubsection{Want to have}
\begin{itemize}
	\item System works offline
\end{itemize}
