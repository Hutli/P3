\label{interviews}
The interviews described in this section are conducted on various bars on the well known party street, Jomfru Ane Gade, in Aalborg. The participants includes staff from \enquote{Fabrikken}, \enquote{LA bar}, \enquote{Newcastle} and \enquote{Dan's Poolhall}. The interviews are semi-structured with questions like the following:

\begin{itemize}
  \item How do you manage your music system?
  \item Are song wishes from the customers granted? If so, how much time is spent on song wishes?
  \item Have you used any other systems? Pros and Cons
  \item Do you have genre/theme nights?
\end{itemize}

If some interesting answers to these questions were given, the interviewer would follow up on these answers with some improvised questions. Hence the semi-structued nature of the interviews.

\subsection{Fabrikken}
\label{sub:fabrikken}

The interviewee is the owner of the bar. Currently the music system in use is a computer running Spotify with a premium license. Customers have influence regarding the music played, but the songs requested have to be approved by the staff. This is to avoid sudden style in the music, which could ruin the party mood for other customers.

Most of the customers in the bar are university students. Therefore the owner considers setting up a Wi-Fi network so that during the daytime, when there is a calm mood in the bar, it enables a suitable environment for studying, very much like a café environment.

\subsection{LA bar}
\label{sub:la_bar}

The interviewee is a bartender. The bar uses Spotify at daytime. The bartender does not need to maintain the Spotify music system other than switching songs upon acceptable requests from customers. The playlists being used are prepopulated by the owner of the bar.

The age group of the customers is 20-30 years, but sometimes married couples also enjoy having a beer in the afternoon.

At night when the party mood is on the rise, a dedicated DJ manages the music.

\subsection{Newcastle}
\label{sub:newcastle}

The interviewee is a bartender. The music system in use is MiB Pro, which runs on a computer behind the bar desk. Because of this, only the bartenders can control the system. The playlists being played from are prepopulated by the owner of the bar. When asked on problems with the system, it was pointed out that there is a little shortage in the music repertoire. Customers can, like in the other bars, influence the music by requesting songs as long as the songs fits in the overall theme of the bar.

Similar to LA bar, a DJ is administrating the music every night.

When asked on an opinion of the features of the system described in this report, the concept was received well, but some issues were raised. The software system would only be used outside of the DJ playtime at night, before the crazy party atmosphere arise.

\subsection{Dan's Poolhall}
\label{sub:dan_s_poolhall}

The interviewee is the bar manager. Spotify is their current music system and is used in the same manner as in the other bars. The staff is making their own playlists, and song requests are possible from the customers. A problem pointed out in the use of Spotify is the need for a stable internet connection, in order for acceptable playback. 

When confronted with the features of the software system described in this paper, the overall response was positive, but some problems came to mind. The system would have to avoid interuptions in genres and slow songs being played in the middle of a party.

\subsection{Summary}
\label{sub:summary}

In conclusion of the interviews made with bar, a specific requirement of still having control over what music is being added to the playlist. This is to ensure that the music fits the usual theme or genre of the bar, and there would not be any sudden change in tempo. The system would in most cases be in operation outside periods of DJs playing. A need of a stable internet connection was mentioned, to ensure a continues stream of music, but this is not a revelant problem \frnote{why isn't it a revelant problem?} if the bar is already using spotify \footnote{A internet music streaming service}, which was the case in 3/4 of the instances.
