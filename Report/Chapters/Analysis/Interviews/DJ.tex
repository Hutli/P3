\section{DJ interview}
One of the important parts from the interview with the pubs was that continuity in the music was very important. So we wanted to get information on how to structure the night at a public place. A DJ has a lot of experience managing music at different events. Thus an interview with DJ Morten Morville was conducted~\cite{int_dj}. Morten is a danish DJ that do gigs at clubs and private parties.

\frnote{noget med metoden til interviewet}

%Some things we got out of the interview with the DJ
The DJ buys music digitally from online music stores like iTunes \footnote{\url{www.apple.com/uk/itunes/}} or beatport \footnote{\url{www.beatport.com/}}. Digital music is easy to carry around, therefore he is able to bring all his music with him all the time. This means he does not have to decide want music to bring ahead of an event. 

When building the structure of a night the DJ usually starts with uncommercial \frnote{er det den bedste oversætning af ukommercielle} music and then later at prime time he will play more mainstream music. At the end of a nigh he likes playing music that can get people going even more.

When playing at club he is more strict with wishes since he is responsible for keeping the music consistent and keeping everyone happy. At private parties he has to be open to wishes.

He is not interested in software to assist him in choosing music or recording the playlist for a night and given feedback.

\frnote{hvordan skal vi bruge det, gider vi selv lave et}
\frnote{hvordan skal vi lægge det med rapporten?}
\kanote{referer til anden gruppen}