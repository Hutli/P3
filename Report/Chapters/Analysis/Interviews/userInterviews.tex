\subsection{Interview of users}
\label{userInterviews}

There exists two types of end users for the product, the bar manager which administrates the service and the customers at the bar. The customers are allowed to affect the music at the bar, by use of the application. So to understand which type of persons that could be interested in using the application, some interviews of customers were performed.

The interview was conducted in the same semi-structured nature as described in \cref{sub:procedure}. The questions for the interview can be found in \cref{app:interviewguidePotentielleBrugere_guide}. In order to find the right participants for the interview, the interviews were performed with people at representative locations in Aalborg. It was chosen to do interviews at these 4 locations in Aalborg:

\begin{itemize}
    \item \emph{BasisBaren} the biggest bar at the university of Aalborg.
    \item \emph{West end} a bar located slightly away from center of Aalborg.
    \item \emph{Baresso} a coffee shop in the center of Aalborg.
    \item \emph{Studenterhustet} a bar and social meeting place for students of Aalborg university.
\end{itemize}

Six interviews were done with a total of 12 Customers, which means some of the interviews was performed on a group of people. All the participants were chosen randomly.

All the participants that were interview was between 19 and 26, years old. Most of the people said it was a good idea to implement a system which could adjust the music at bars to cater to the customers music preferences. There were some disagreement between the participants in how the system should determine the next song to be played. There were a tendency to the slightly older participants, did not actually listen to the songs that were played. All agreed that there has to be music and the 19 to 21 years old more often requested songs compared to the older audience.
