There are generally two ways to interact with a system. In this project, they are defined as centralized or decentralized interaction.

Centralized interaction means that communication has to flow through a single point.

Decentralized interaction means that communication is not dependent on a single point, but instead is possible from multiple points.

% This project involves how users can interact with the music being played at a venue e.g. request tracks to be played. As presented in the introduction, most venues currently rely on a centralized interaction form. In most cases this is bartenders or waiters working at the venue. Venues require this form of centralized interaction because of the need to maintain control of the music being played. However by taking this centralized approach to interaction, the bartenders or waiters need to dismiss time from their primary tasks to interact with users.

% This project involves the interaction at social venues and how people can interact and have influence on what the music that is played at the venue. As presented in the introduction most venues currently rely on a centralized human interaction with the people serving at the venues, for instance bartenders or waiters, to be the ones taking requests from costumers and combining this with the venues interests to decide what music is to be played. As already mentioned this is not optimal since this is not his or her primary function. This can result in people not getting the desired influence or the waiter/bartender to complete his or her primary task unsatisfying.

When working with user interaction there are two main considerations. How the user recives data (input) and how the user influences the system (output). The input the user gets can either be centralized e.g. a monitor at a central location, or decentralized e.g. the user's smartphone showing the input.

The user can also influence the system either centralized e.g. by communicating with venue workers, or decentralized e.g. by interacting with the user's smartphone.

Systems can mix and match their interaction forms i.e. decentralized from the user to the system, but centralized from the system to the user.

% When working with user interaction there are two main considerations. How the user recives data (input) and how the user influences the system (output).
% Currently the user can interact with the system via a centralized system where all users can interact at one or few interaction points. The alternative to this is a decentralized interaction point system where each user can interact with the system via individual interaction points.
% The same two possibilities goes for the input to the user. Meaning that the input, the user being able to see the result of his or her interaction, can either be centralized combining the interaction from all the users in one place and displaying it there or decentralized, showing the result for his or her interaction and possibly the combined interaction from all too.

% Furthermore both the input and output interaction points can be combined meaning that the system could either have a completely centralized interaction point or completely decentralized interaction points enabling the users to both interact with the system and recieve conformation at the same point.
