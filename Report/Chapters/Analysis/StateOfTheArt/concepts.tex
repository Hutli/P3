When analysing the state of the art it was found that many different concepts was being used in the basis of the design of the interactive system to enabling the users influence on the venues music. Three systems were choosen, SecretDJ, Mixgar and Rockbot, since they togeather represent all the different concepts found during the research and illustrates the implementation of these.

The rest of this section will focus on describing each concept and its specific implementation.

\subsection{Voting}
All three systems uses voting in some way. SecretDJ and Rockbot only uses up-votes, or likes as SecretDj calls it, meaning that in these systems, people can only vote on songs they like and want to be played sooner. Mixgar also adapt the concept of down-voting songs enabling the user to vote song they don't want to hear (or not yet want to hear) down the play-list making them getting played later or not at all.

\subsection{Request}
SecretDJ - A user can initially add 4 songs each day
Mixgar - Indirect and automatic song add form the the party's mood curve, votes and individual preferences.
Rockbot - Unlimited requests but the higher level the higher on the playlist the song gets.

\subsection{Play-list}
All three systems work with the concetp of a playlist. Songs are added to the playlist via requests and move up and down via votes and/or the automated system \enquote{deciding} what is most fitting for the vanue at its current state.

\subsection{Social network}


\subsection{Automatic rearrangement of playlist}
Mixgar works with the concept of the system adding songs to the playlist automatically. First of the system uterlizes the users’ Facebook likes to find out what artists and genres the users at the venue likes. With this data, a playlist made from Youtube’s database is created. After this the system accumulates data about the party’s mood curve, the progression of songs that are played, and can automatically arrange songs in the playlist if the
administrator chooses to switch to what they call autopilot mode.

\subsection{Editing of playlist}


\subsection{Restriction of playlist}


\subsection{Levelling}
Voting can also be used, for instance in the case of SecretDJ, to rise the level of another user, explained later, thereby enabeling the user to let his or her songs start higher on the playlist of the specific venue.

\subsection{Check-in}


\subsection{All time available venue information}
SecretDJ -  "The app gives its users the ability to see which venues use the SecretDJ system, and which are nearest to the user."


\subsection{Indirect song request and voting}
In the case of Mixgar the users cannot request songs and their votes does not directly and completely decide the flow of the music. 