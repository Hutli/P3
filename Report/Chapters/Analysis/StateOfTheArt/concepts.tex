When analysing the state of the art it was found that many different
concepts were being used in interactive music systems. Three systems
were chosen, SecretDJ, Mixgar and Rockbot, since they together
represent many concepts found during research and illustrates the
implementation of these.

The rest of this section will focus on describing each concept and its
specific implementations. These concepts serve as inspiration later in
the design phase of this project.

\subsection{Playlist}
All three systems work with the concept of a playlist. Tracks are
added to the playlist by user requests. The tracks can now change
positions in the queue in two ways. The users can vote on the tracks
or the music system can decide what the most fitting track is at any 
given time.

\subsection{Voting}
All three systems implement voting of tracks in some way. SecretDJ and
Rockbot only allow up-votes i.e. users can only vote for a track to be
played sooner. Mixgar also implement the concept of down-voting
tracks, enabling the user to vote for tracks to have their playback
postponed.

\subsection{Requesting}
The three systems each implement requesting in different ways. The
first implementation method of the concept, done by Rockbot, is
unlimited requests.  Any user can add unlimited tracks to the
playlist. To balance this, requests made by more active and liked
users, start in better positions.

The second approach is that users have a limited amount of
requests. This is done by SecretDJ where the limit initially starts at
four tracks per user per day and then rises as the user becomes more
active and liked.

The third and last implementation is done by Mixgar. The playlist is
automatically rearranged by the system. This approach is described in
greater detail in \cref{sub:auto_rearrange_playlist}.

\subsection{Social Network Integration}
SecretDJ and Mixgar utilises social networks to better target the
music to the users by analysing their music preferences. The
integration of social networks also allow social interaction between
the users of the system based on their votes.

\subsection{Automatic Rearrangement of Playlist}
\label{sub:auto_rearrange_playlist}
In the case of Mixgar the users cannot request tracks and their votes
does not directly and completely decide the flow of the music. Mixgar
works with the concept of the system adding tracks to the playlist
automatically. First off the system utilises the users’ Facebook likes
to find out what artists and genres the users like. With this data, a
playlist is created. The system now accumulates data about the party’s
mood curve from the users votes and the progression of tracks that are played. The system
can then, based upon this, automatically arrange tracks in the
playlist.

\subsection{Administrative Editing of Playlist}
Both Rockbot and
Mixgar implements the ability for administrators at venues to remove, add
and move tracks around on the playlist, essentially overriding user choice. Especially for Rockbot, this
is a very strong selling point for businesses which suggests that it
is very important for the venues to have the last say and control in
what is played since it is their bar and neither the system nor the users
should be able subdue the venues' wishes.

\subsection{Administrative Restriction of Playlist}
In both Mixgar and Rockbot's system the administrators are able to
restrict what users are able to add to the playlist. This means that
the administrators of the venues can decide which genres, artists,
albums etc. are allowed or disallowed to be added by the users.

\subsection{User Ranking}
SecretDJ and Rockbot implements the concept of an user having a rank or
level. This rank is determined by activity in the system or by other
users' votes. Higher ranked users have benefits such as higher degrees
of influence on the music being played.

\subsection{Check-in}
Rockbot implements the concept of checking-in at a venue. Checking-in
can only be performed when in close proximity to the venue. When and
only when checked-in the user can interact with the music system.

\subsection{All Time Available Venue Information}
SecretDJ gives its users the ability to list all the venues that are
using the SecretDJ system. The list includes general information on
the bar, distance to the user and what kind of music the venue is
playing.

%%% Local Variables:
%%% mode: latex
%%% TeX-master: "../../../master"
%%% End:
