\chapter{Analysis}

This section analyses the initial problem described in the introduction. The analysis methods used originates from \cite{mathiassen2001objektorienteret}. 

\section{Problem Domain \& Application Domain}

\sinote{Dette afsnit er design orienteret. Er ikke færdigt}

\textbf{Problem domain}: \enquote{That part of a context that is administrated, monitored or controlled by a system.}

\begin{description}
  \item[Songs] The songs that are available for playing.
  % \item[Playlists] Collections of songs centered around a specific theme.
  \item[Votes] Some mechanism for choosing which songs to play next.
  \item[Users] Data and statistics of the users of the system.
  \item[Places] What is playing at particular places?
  \item[Audio system] Plays the chosen song.
  \item[Display system] Displays queue, current song, etc. at the installation site.
\end{description}

\textbf{Application domain}: \enquote{The organization that administrates, monitors, or controls a problem domain.}

\begin{description}
  \item[Users] The users of the application chooses which songs to play next.
  \item[Frontend] Provides control of songs, playlists and users.
  \item[Backend] Administrates and monitors users of the application. Communicate with the audio- and display system.
\end{description}

\section{Architecture}
\label{sec:architecture}

\sinote{Dette afsnit er design orienteret. Er ikke færdigt}

Model:

Songs:
\begin{itemize}
  \item Current song
  \item Queue
  \item Song repertoire
\end{itemize}

Votes:
\begin{itemize}
  \item Upvotes
  \item Downvotes
  \item Master override
\end{itemize}

Users:
\begin{itemize}
  \item Votes statistics. Affects user's influence
\end{itemize}

SecretDJ gives its users the ability to see which venues use the SecretDJ system, and which are nearest to the user. The users can add songs to the venue's playlist and give likes, via Facebook, to the songs, moving them up the playlist.\\

A new user can add four songs to a venue's playlist each day. If a user gets a lot of likes for the songs he has added to a venue's playlist, the user can rise in rank for that venue, which means the user can add more songs to that venue's playlist and that his choices are moved further up the playlist by default. 
If a user gets enough likes and is active enough, the user can rise to the rank of DJ for a venue, meaning that the user's choice of songs will always be next in line, and there is no limit to the amount of songs the user can add to the playlist.
