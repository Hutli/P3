\subsubsection{Client Application}
\label{ssub:client_application}

When designing a client application many things have to be considered. In which settings are the application going to be used? How are the users going to interact with the application? \sinote{flere eksempler?}. The following paragraphs list the choices made when creating a client application.

\paragraph{Native Application or Web Application}
\label{par:native_application_or_web_application}

As described in \sinote{reference til sted i analysen hvor context bliver beskrevet} the client application is used in bars and pubs. Therefore the application has to run on a mobile platform. As the mobile platform is in rapid growth new possibilities appear and disappear quickly. However smartphones have become common property. Applications on smartphones can either be native applications that are installed on the device or be web applications that are accessed through the device's browser. The key differences between the two types of applications can be summarised by; native applications often feel more responsive and look more integrated to the smartphone environment. On the downside native applications have to be installed on the device, taking time to install and filling up the smarthone's memory. Web applications on the other hand are ready in an instant.

As the responsiveness of the application is important for a pleasant user experience, it is chosen to develop the client application as a native application. By making it native to the platform, the look and feel of the application is more familiar to the user because user interface elements share the same traits as the rest of the well-known platform.

\paragraph{Native Application Development Framework}
\label{par:native_application_development_framework}

Many native application development frameworks exist, including Android SDK, iOS SDK, Windows Phone SDK, Qt, Xamarin and more. They each have their different characteristics. For example the Android SDK makes it possible to build good looking, native Android applications. The applications are written in the programming language Java. Another example is Qt. Qt is cross platform, so the application written using Qt can run on a variety of systems including Android, iOS and Windows Phone. Qt is written in programming language C++.

With so many good frameworks it is hard to choose one to go with. Fortunately the semester description helps to decide by specifying that all programming should be in C\#. This narrows the possibilities of frameworks down to just Xamarin.

Xamarin makes it possible to write an application in C\# and run it on iOS, Android and Windows Phone. The application will look differently on the three platforms, but they will feel native on each of the platforms.

Because Xamarin enables native applications for all three large mobile platforms, the client application can reach a large userbase. The client application is therefore chosen to be written using the Xamarin framework.


