Naming:

Naming a system is creative process, and therefore subjective. Methods for creative thinking were applied for finding the collective most subjectively fitting name for the system. First by a brainstorming session, resulting in a list of word related to music, democracy and social activities, the earlier analysis in state of the art contributed in supplying various buzzword and word endings, -ify.

This being a open source project, an emphasis on this fact were pleasing for the project group, with a good reasoning by the market for open source software becoming increasingly popular (source).

Logo:

From the brainstorming session of trying a fitting name, ideas drawn, balance is symbolised as the \enquote{yin yang} \cref{fig:yinyang} symbol and the \enquote{next song} button from a radios interface \cref{fig:vlc} is symbolising the playlist aspect of the system.

\begin{figure}
  \centering
  \includegraphics[width=0.5\linewidth]{Images/vlc.jpg}
  \caption{VLC player}
  \label{fig:vlc}
\end{figure}

\begin{figure}
  \centering
  \includegraphics[width=0.5\linewidth]{Images/Yin-Yang.png}
  \caption{A symbol of balance by !!SOME standards}
  \label{fig:yinyang}
\end{figure}

An effort to fusion these to values, the green color is carried over from the logo of platform the system is developed upon \enquote{Spotify}.

\begin{figure}
  \centering
  \includegraphics[width=0.5\linewidth]{Images/Icon.png}
  \caption{The resulting logo of the system}
  \label{fig:logo}
\end{figure}