\section{Implementing the Administrator Interface}\label{sec:impinterface}
As stated in \cref{sec:serverInterface}, the usability of the server interface was not considered. Therefore, this section focuses on explaining how previously described functionality is implemented and can be interacted with by the administrator.

The first screen the administrator meets, when opening the application is the login screen for Spotify, as seen in \cref{fig:loginInterface}. On this screen, the administrator inputs the login credentials for the venues' Spotify account. From this interface, the administrators can stop and start the playback of music, as described in \cref{systemDefinition}. The icon for the application acts as a skip button, where administrators can skip the currently playing track.

\begin{figure}[H]
  \centering
  \subfloat{
    \includegraphics[width=0.5\textwidth]{ServerInterfaceLogin}
  }
  \subfloat{
    \includegraphics[width=0.5\textwidth]{ServerInterfaceLoggedin}
  }
  \caption{Login interface on the server.}\label{fig:loginInterface}
\end{figure}

The second interface is the playlist, which can be seen in \cref{fig:ServerInterfacePlaylist}. On this screen, the most crucial information about tracks, on the playlist, can be seen; the title, artist(s), duration (in milliseconds), temporary and permanent votes. The album cover is also displayed for the ability to quickly identify tracks. The playlist is ordered in the order they will be played, the higher a placement, the sooner it will be played. From this interface, further administrative control can be executed, this includes removal of specific tracks and rearrangement of the playlist, as described in \cref{systemDefinition}. When pressing the move up or move down button the currently selected track changes one position, this is done through changing the permanent votes just enough for the track to change place.

\begin{figure}[hbtp]
  \centering
  \includegraphics[width=\textwidth]{Images/ServerInterfacePlaylist.png}
  \caption{Playlist interface on the server.}\label{fig:ServerInterfacePlaylist}
\end{figure}

The third interface is the history interface, seen in \cref{fig:ServerInterfaceHistory}. From here, the administrator can see what has been played previously. The higher on the list, the longer ago it was played. Like the playlist interface, only the most crucial information is displayed. 

\begin{figure}[hbtp]
  \centering
  \includegraphics[width=\textwidth]{Images/ServerInterfaceHistory.png}
  \caption{History interface on the server.}\label{fig:ServerInterfaceHistory}
\end{figure}

The fourth interface is the restriction interface, which can be seen in \cref{fig:ServerInterfaceRestrictions1}. From here, the administrator can modify, remove and add restrictions. Adding or modifying a restriction brings up a dialogue, where parameters for the restriction are as described in \cref{sec:restrictions}.

\begin{figure}[H]
  \centering
  \subfloat[Main restriction interface.]{
    \includegraphics[height=160px]{ServerInterfaceRestrictions}
        \label{fig:ServerInterfaceRestrictions1}
  }
  \subfloat[Restriction modify dialogue.]{
    \includegraphics[height=160px]{ServerInterfaceRestrictionDialog}
        \label{fig:ServerInterfaceRestrictions2}
  }
  \caption{Restriction interface on the server.}
\end{figure}

The last interface is the user interface, which can be seen in \cref{fig:ServerInterfaceUsers}. From here, the administrator can keep track of how many and which guests have checked in at the venue. The administrator can see what they have voted for, both track and volume. The guests are represented via their unique ID, as described in \cref{uid}. As this representation is nonsensical to the administrator, alternatives to this approach will be discussed in \cref{UAuth}.


\begin{figure}[hbtp]
  \centering
  \includegraphics[width=\textwidth]{Images/ServerInterfaceUsers.png}
  \caption{User interface on the server.}\label{fig:ServerInterfaceUsers}
\end{figure}
