\chapter{Design Patterns}

\section{Model View Viewmodel}

This project is both on the client and on the server structured using
the Model View ViewModel (MVVM) pattern. The core idea in MVVM is to
separate the presentation layer, the view in MVVM, from the model
layer, the model in MVVM. This separation is done by introducing a
ViewModel layer. The ViewModel exposes data from the model to the view
in such a way that the view layer never has any knowledge about the
model layer.

This pattern is used mainly because the frameworks used on the client
and server are designed to be structured using this pattern. Other
similar patterns such as Model View Controller (MVC) could
theoretically also have been used, but doing so would go against some
of the principles the frameworks used are based upon.

\section{Singleton Pattern}

The singleton pattern is used whenever there should only ever be one
instance of a class. The pattern is often implemented as a static
method on the class, that instantiates the class on the first call,
and on later calls returns the created instance of the class.

This pattern is used in SpotifyDotNet, the libspotify wrapper, used in
this project. Because it is only possible to be logged in one user at
a time, the SpotifyLoggedIn class implements the singleton pattern.

\section{Dependency Injection}

Dependency Injection is a design pattern that helps to reduce coupling
between different components in a software program. The main idea is
to have components not depend on concrete dependencies but instead
depend on abstractions of dependencies. Doing this, dependencies can
be dynamically changed during runtime as long as their abstractions
match up.

In this project the StructureMap framework is used to reduce
boilerplate and ease the implementation of Dependency Injection.
