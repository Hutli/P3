\section{White- and Blacklist}
As described in \cref{sec:restrictions} the system should implement a form of restriction though white and black listing. An obvious implementation is to whitelist and blacklist specific tracks in the music catalogue. Making such a filter could be a very tedious process with a very large music catalogue. Somehow easing the process of making restrictions would be beneficial for the administrator, this could be implemented by allowing for creation of coarser restrictions. The coarser restrictions are based on metadata related to the tracks, beside the title; the artist of the track, the genre of the track and some signature tags, like "fast pace" or "electronic". Spotify provides very limited metadata in their database of music, but does include the most significant ones such as, what album the track is on, the artists who made the track, if it is explicit, the length of the track and occasionally the genre and popularity. Further tags and metadata could be provided by a third party database, such as Musicbrainz. An implementation of another provider of metadata, was concluded to be outside of the scope of the project, as metadata is available through Spotify. To avoid that the same track is found across both databases, an International Standard Recording Code (ISRC) is provided from Spotify’s database. This can be used to uniquely identifying a music track across multiple databases \cite{isrc}.

The restrictions are implemented as a list of predicates, that each individual track must meet in order to be added to the playlist. These predicates return boolean value indicating whether a track is allowed e.g. by evaluating the artists of the track, if a blacklisted artist occurs on the track. When the user is searching for a track, it will be marked as not available to explicitly state that the track was indeed found in the music catalogue, but is not allowed to be played at the specific venue. This is done in order to minimize any confusion during the search phase.
