\section{Whitelist and Blacklist}

An obvious implementation of these concepts is to whitelist and blacklist specific tracks, this is a solution but to ease the process of making such extensive restriction, for the administrator, a possibility of making more coarse restrictions would be beneficial. These restrictions is based on metadata related to the tracks; the artist of the track, the genre of the track and some signature tags. Spotify provides the artist and occasionally the genre. The tags cloud be provided by another third party database of tags. To ensure that the same track is found across both databases, an International Standard Recording Code (ISRC) is provided from Spotify’s database. This can be used to uniquely identifying a music track \cite{isrc}.

This is implemented as a list of predicates, that each individual track must meet in order to be added to the playlist. These predicates add to evaluate the artists of the track, if a blacklisted artist occurs on the track, the track is discarded, before being added. When the user is searching for a track, it will be marked as \enquote{filteredOut} to explicitly state that the track were found but is not allowed to played at the specific venue This is done in order to minimize any confusions made during the search phase.
