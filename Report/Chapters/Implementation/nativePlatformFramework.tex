\section{Native Application Developement Framework}
\label{par:native_application_development_framework}

Many native application development frameworks exist, including Android SDK, iOS SDK, Windows Phone SDK, Qt, Xamarin.Forms and more. They each have their different characteristics. Android SDK makes it possible to build Android applications. The applications are written in the programming language Java. Similarly you can use iOS SDK to build iOS applications and Windows Phone SDK to build Windows Phone applications using Objective C or Swift, and C\# respectively. Qt is cross platform, so the application written using Qt can run on a variety of systems including Android, iOS and Windows Phone. Qt is written in the programming language C++. Xamarin.Forms is also cross platform and is written in C\#.

With so many good frameworks it is hard to choose one to go with. Fortunately the semester description helps to decide by specifying that all programming should be in C\#. This narrows the possibilities of frameworks down to Xamarin.Forms or limiting the platforms to only Windows Phone devices.

Limiting the application to only Windows Phone devices would drastically limit the amount of devices which could run the application.

Xamarin.Forms makes it possible to write an application in C\# and run it on iOS, Android and Windows Phone. The application will look differently on the three platforms, but they will feel native on each of the platforms.

Because Xamarin.Forms enables native applications for all three large mobile platforms, the client application can reach a large user base. The client application is therefore chosen to be written using the Xamarin.Forms framework.

\section{User Identification}
\label{uid}
By using a device constant like the IMEI number, associated with a smartphone, the authorisation is automatic so the user does not need to do any login. This greatly simplifies the authorisation both for the user and for the developer. A limitation of this approach is that all user data associated with the device constant is lost if the user buys a new device, and thereby uses a new device constant. Therefore, it was chosen to use the IMEI number as user identification. %%% Local Variables:
%%% mode: latex
%%% TeX-master: "../../master"
%%% End:
