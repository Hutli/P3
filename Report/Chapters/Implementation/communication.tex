\paragraph{Communication with client}
As the user has to find the track they wish to hear on their mobile device and the track has to be played on another device a means of communicating the request from one device to another had to be found.

For communicating the filter and votes between the server and client, Nancy\footnote{http://nancyfx.org/}, a lightweight framework for build http services on .Net, available from NuGet, was chosen. Nancy is easy to set up as self-hosting. It can run asynchronously, only reporting back to the main thread when a request is received. The client simply has to send a http request on a designated port, to interact with the server\dots \chnote{Skal udvides vi har bare ikke rigtig undersøgt muligheder og begrænsninger}

\paragraph{Audio output}
When receiving the data stream from Spotify, some way of outputting the audio is needed. The data stream is of type Pulse Code Modulation (PCM). In searching for a way to playback PCM data it was found that Naudio\footnote{http://naudio.codeplex.com/} was most sensible choice in terms of extensibility.

%When receiving the data stream from Spotify, some way of outputting the actual audio is needed, the data stream is of type Pulse Code Modulation (PCM), in searching for a way to playback PCM data the following candidates were found:
%\begin{itemize}
%  \item Naudio~\cite{naudio} - A well established open source library, still being updated, available through NuGet
%  \item Alvas.Audio~\cite{alvas} - Proprietary C\# audio library for .Net. Requires a license.
%  \item BASS.NET~\cite{bass} - A 3rd party .Net wrapper for the BASS audio library, still being updated. Requires a license.
%  \item SoundPlayer - Intergrated in the .Net framework, but only supports .wav PCM data stream.
%\end{itemize}
%The features of these candidates is pretty distributed, apart from the SoundPlayer class of the .Net framework, Should the system be required to handle other compressed data streams, SoundPlayer would struggle due to having to convert the data before delivering it. Of the remaining candidates the project group sought to minimize cost of the system, and keep it as open source as possible, thereby leaving out the proprietary candidates. Leaving Naudio as the only candidate and most sensible in terms of extensibility.
