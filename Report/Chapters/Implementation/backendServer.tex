\section{Backend Server}
\label{imp:backendServer}

As described in \cref{techPlat:backendServer}, HTTP endpoints must be
available to the client from the server. This section presents all the
available endpoints and describes key implementation
details. Additionally the details of the playback of tracks from
Spotify are detailed.

Nancy, a lightweight framework for building HTTP services in C\# is leveraged
to cut down development time.

\subsection{Checking-in}
For checking in to a venue, the client has to send a unique user ID to
the endpoint shown in \cref{lst:endpoint_checkin}

\begin{lstlisting}[label={lst:endpoint_checkin}, caption={HTTP endpoint allowing client to check-in to a venue. Text surrounded by curly brackets are parameters.}]
http://server:port/checkin/{userID}
\end{lstlisting}

\subsection{Searching}
To search, just a search query is needed.

\begin{lstlisting}[label={lst:endpoint_search}, caption={Text surrounded by curly brackets are parameters.}]
http://server:port/search/{query}
\end{lstlisting}

\subsubsection{Returns}
Tracks on Spotify that matches the query requested and are allowed to
be played at the venue. This is formatted in JSON.

\subsection{Voting}

\begin{lstlisting}[label={lst:endpoint_vote}, caption={Text surrounded by curly brackets are parameters.}]
http://server:port/vote/{userID}/{trackId}
\end{lstlisting}

\subsection{Check-out}

\begin{lstlisting}[label={lst:endpoint_checkout}, caption={Text surrounded by curly brackets are parameters.}]
http://server:port/checkout/{userID}
\end{lstlisting}

%%% Local Variables:
%%% mode: latex
%%% TeX-master: "../../master"
%%% End:
