\chapter{Future Improvements}
\label{future}

This section describes some of the concepts which were excluded from the scope of the project, but could be included in a future system, solving the same problems as presented in this project.

\section{Incorporate Multiple Music Catalogues}
A music catalogue, like Spotify used in this implementation, is not complete with all the music available. To expand the music available to the user, a collective music catalogue of several different music catalogues could be implemented. An important detail on this is that tracks might appear multiple times across these catalogues.

\section{Gather and Utilise Data About Users}
The system could collect statistics about individual users and their voting habits, to be able to provide music recommendations.

To further decrease time spend on interaction with the system, the
users' music related likes and favourites on social media (SoundCloud,
Facebook etc.) could be used to generate playlists. This introduces a
kind of passive interaction between the user and the system; the user
is affecting the selection of music indirectly.

\section{Reward/Progression}
A reward and/or progression programme could be implemented to further encourage users to vote beside the benefit of listening to their favourite tracks. Introducing a reward, like a free drink at the venue or a higher vote influence, i.e. a vote counts as two votes, in the system, would even further motivate users to use the system. Such programme would give the user a sense of progression, which has shown very beneficial in the industry of computer games, in keeping the user actively using the system~\cite{games}. This might however contradict the aim of minimising users' time spent using the system.

\section{Physical Activitity}
Measuring physical activities of the user, would introduce a wide
range of concepts that could be implemented in the system. The system
could include a "activity-weight" in the equation of choosing the next
song to play, for example by reading from the accelerometers of the
mobile devices interacting with the system. A high activity could
suggest that the user is dancing, which could result in the users vote
having a higher weight. Various other sensors could read temperatures
on the dance floor, or density of people for statistics on feedback
on the currently playing track.

\section{Security}
If this solution is to become a standard in its field, security against attacks will have to be implemented, as the likelihood of an attacker to try to exploit the system grows with the amount of users, and people in general who are aware of the system. Any security flaw reaching the end users has the potential to break the fairness of the solution and, more seriously, deteriorate the reputation of the solution, possibly killing any chance of widespread use.

\section{User Authorisation}\label{sec:UAuth}
In order to differentiate users of the client application, some mechanism of authorisation is needed in the client application. Possible mechanisms are: logging in with already established login services like Google login and Facebook login, logging in with a self made login system or basing the authorisation of a device id e.g. IMEI number or a phone number.

The good thing about reusing login services like Facebook login or Google login which are well known is that users do not need to create a new account just to use the application. Another benefit is that security is taken care of externally i.e. secure password storage is not a concern in the development of this application. Some users might have a stigma against these services or fear of their privacy by using the services e.g. that the client application would post on their Facebook timeline.

The only benefits of creating a self made login system is that there is more control about the user's accounts and the users do not have to have fear the application manipulating their Facebook account. However, the benefits from external login systems are lost i.e. password storage would have to be implemented in a secure way and kept up to par with the latest exploits. Users would also have to create an account solely for the application.

By using a device constant like the IMEI number, associated with a smartphone, the authorisation is automatic so the user does not need to do any login. This greatly simplifies the authorisation both for the user and for the developer. A limitation of this approach is that all user data associated with the device constant is lost if the user buys a new device, and thereby uses a new device constant.
