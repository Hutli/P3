\section{Future Improvements}
\label{future}

This section describes some of the concepts, excluded from the scope of the project, but could be included in a future system, solving the same problems as presented in this project. 

\subsection{Incorporate Multiple Music Catalogues}
A music catalogue, like Spotify used in this implementation, is not complete with all the music available. To expand the music available to the user, a collective music catalogue of several different music catalogues could be implemented. An important detail on this is that tracks might appear multiple times across these catalogues.

\subsection{Gather and Utilise Data About Users}
The system could collect statistics about individual users and their voting habits, to be able to provide music recommendations.

To further decrease time spend on interaction with the system, the users' music related likes and favourites on social medias (SoundCloud, Facebook etc.) could be used to generate playlists. This introduces a kind of passive interaction between the user and the system; the user is affecting the music to be played indirectly.

\subsection{Reward/Progression}
A reward and/or progression programme could be implemented to further encourage users to vote beside the benefit of listening to their favourite tracks. Introducing a reward, like a free drink at the venue or a higher vote influence, i.e. a vote counts as two votes, in the system, would even further motivate users to use the system. Such programme would give the user a sense of progression, which has shown very beneficial in the industry of computer games~\cite{games}, in keeping the user actively using the system. This might however contradict the aim of minimising users' time spent using the system.

\paragraph{Physical activitity}
Measuring physical activities of the user, would introduce a wide range of concepts that could be implemented in the system. The system could include a "activity-weight" in the equation of choosing the next song to play, for example by reading from the accelerometers of the mobile devices interacting with the system. A high activity could suggest that the user is dancing, which could result in the users vote having a higher weight. Various other sensor could read tempatures on the dancefloor, or density of people for statistics on eg. feedback on the currently playing track.