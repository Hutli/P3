Mixgar utilizes the users' Facebook likes to find out what artists and genres the users like. With this data, a playlist made from Youtube's database is created for the party, where people can up- or downvote tracks.\\

The system accumulates data about the party's mood curve, the progression of tracks that are played, and can automatically arrange tracks in the playlist if the administrator chooses to switch to what they call autopilot mode.
The administrator can choose to skip tracks, remove them from the playlist or restrict the genres that are allowed to be played.

The concept of this system is that is takes already known data about the users at a party to make a playlist, that the users can then mutate, to fit the mood of the party. A problem with this is that genres and themes is set by the users not the host, which was conducted from the interviews of the bars to be a requirement. Though it is still possible to have some what control over the music by being able skip or remove tracks from the playlist, and restricting the genres allowed to be played. The way the users interact with playlist is by upvoting and downvoting on specific tracks on the playlist, which means the indivuals users have their saying on what is being played to a limit, one can affect arrangement of the tracks on the playlist by upvoting the tracks, one wants to listen to sooner, or downvoting the tracks one does prefer or wants it to be played later. The concept of downvoting may be potential problem or a potential benefit by cliques of the people downvoting the same track, which may cause the track never to be played, which may introduce a sense of competition between cliques, but may also just introduce annoyance for users of the system by not being able to hear their preferences, which could lead to users not wanting to use the system, which would be harmful to the systems reputation. The effect may also apply to upvoting at a different scale.
