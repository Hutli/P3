From the system definition, some entities can be found that need to be modelled in the system's problem domain, in order to work with them in the application domain. These are the candidate entities for the classes that the system needs:
\begin{description}
    \item[Track]
    A track class for metadata e.g. title, artist, duration. With this information a vote made by a user, can be identified as track.
    \item[User]
    For identifying every individual who wishes to influence the playlist, by vote on some track
    \item[Vote]
    This class is for modelling what track a user voted for to be played, and when the vote was submitted.
    \item[Playlist]
    A collective of track in some order of what and when to play track.
    \item[Restriction]
    This is to allow for a limitation set by the administration, e.g. some of the set of tracks that are allowed to be added to the playlist.
    \item[Venue]
    Is the venue that has a playlist and some restrictions to that specific playlist.
    \item[History]
    To contain tracks that were played in the past. This is this
    relevant to analyse the music flow of the music at the venue.
\end{description}
