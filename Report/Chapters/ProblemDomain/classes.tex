These are the candidates for classes, the system needs to keep track of:
\begin{description}
    \item[Track]

    Used for storing meta data eg. artist, duration
    \item[User]

    Used for information about the user eg. votes
    \item[Vote]

    Which track a user voted for to be played, and when the vote was submitted.
    \item[Playlist]

    Used for keeping track of the votes and tracks
    \item[Restriction]

    To allow limitation of the set of tracks that can be added to the playlist.
    \item[Venue]

    To store information about the playlist and restrictions.
    \item[History]
    To contain tracks that were played in the past.
\end{description}

In order for the user to be able to vote for a track, a uniform way of referencing tracks is needed. This is done though a vote class that also contains the information about the vote; the user that voted, the track voted for and the time is was created. To achieve this, a catalogue is introduced. The user can then browse for a desired track.
