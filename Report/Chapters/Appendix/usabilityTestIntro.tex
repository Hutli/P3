\chapter{Introduktion til brugbarhedstest med brugere}
\label{usabilityTestIntro}

Dette er introduktionen der blev læst op til hver bruger inden de gennemførte testen. Teksten er baseret på en lignende tekst fra en gæsteforelæsning af Jesper Kjeldskov der blev holdt den 2. oktober 2014 some en del af kurset \enquote{Design and Evaluation of User Interfaces}.

Jeg vil gerne starte med at takke dig for, at du vil hjælpe os med at gennemføre denne test, og jeg vil gøre opmærksom på, at det naturligvis er systemet, vi ønsker at teste og ikke dig.

Før testen skal jeg sige et par ting. For at være sikker på, at jeg husker det hele og får det sagt på samme måde til alle deltagere, vil jeg læse det op.

Det system, du skal hjælpe os med at teste, er vores semesterprojekt openPlaylist. Testen forløber på den måde, at du får stillet et antal opgaver, som du skal forsøge at løse ved hjælp af systemet. Opgaverne afspejler en bargængers typisk brug af systemet. Du vil få udleveret opgaverne en efter en.

Vi vil bede dig løse opgaverne på følgende måde. Først læser du hele teksten til opgaven højt. Derefter fortæller du mig, hvordan du forstår opgaven, og hvad du vil gøre for at løse den. Så går du i gang med selve løsningen. Mens du arbejder med opgaven vil vi gerne have, at du tænker højt. Det betyder, at du skal sige, hvad du har tænkt dig at gøre, hvad der overrasker dig og hvad du ellers tænker på under brugen. Jeg ved godt, at det ikke er naturligt at sidde og snakke højt, så hvis du glemmer det, vil jeg minde dig om det.

Hvis du får problemer under løsningen af opgaven, kan du godt spørge mig. Men jeg vil nok ikke hjælpe dig direkte. Jeg vil i stedet forsøge at få dig til selv at komme videre.

Når du mener, at du er færdig med at løse en opgave, vil jeg bede dig sige det, så vi ikke er i tvivl bagefter.

Før testen starter, vil jeg bede dig om at underskrive denne samtykkeerklæring for at sikre, at du er indforstået med rammerne for testen.