\label{interviews}

Understanding the people involved in the system is key when doing requirements analysis. It is important to get an understanding of current workflows and problems associated with these. One technique towards understanding is interviewing. Interviews will be used in this project to gain understanding. The following section describes how the interview was conducted and the results that were collected.

\subsection{Procedure}
\label{sub:procedure}

Interviews can be structured, unstructured or anything in between~\cite{benyon2013designing}. Structured interviews are very strict in their form. Every question is pre-made and the interviewer follows the questions strictly. This reduces the effort needed by the interviewer. However this type of interviews should only be used when the answers given by the interviewee are simple and can fall into discrete categories.

On the other end, unstructured interviews allow for a high degree of flexibility. This form of interviews are good when knowledge in a particular field is not easily accessible. This flexibility comes with the need for the interviewee to be able to generate questions on the fly.

The interviews described in this section are all semi-structured i.e. most questions are written beforehand but the interviewer is not strictly bound to these and can therefore drill into particular answers given, by following up with related improvised questions. Hence the semi-structured nature of the interviews.

The questions are based in the following categories:

\begin{itemize}
  \item Management of bar music systems
  \item Handling requests from the guests
  \item Current and former music systems
  \item Music requirements and the dynamics thereof
\end{itemize}

The full interview guide can be found in \cref{app:interview_guide}. The interviews are conducted by two persons. One is the interviewer and the other one is taking notes and managing the voice recording.

\subsection{Participants}

An important choice in interviewing is who to speak with. There exists two types of end users for the product, the bar manager or bartender, who administrates services and protects the interests of the bar, and the guests who visit the bar. The first interviews conducted where focused on the administers. 

As this report focuses on the context of bars, the participants would have to have experience working at a bar and using the current music system at the bar. Therefore the sensible choice is to find participants in bars. There are different types of bars: sports bars, bars with DJs at night and bars with a pub atmosphere.

It was chosen to speak with four bars, to get a good representative view of most bars. The following bars were chosen since it will cover most types of bars:

\begin{itemize}
  \item Fabrikken
  \item LA Bar
  \item Newcastle
  \item Dan's Poolhall
\end{itemize}

\subsection{Data Collection}
\label{sub:data_collection}

During the interviews the participants were voice recorded and the designated interview helper was taking notes.

\subsection{Interview Analysis}
\label{sub:interview_analysis}

After all the interviews were performed, the group collaboratively created a summary of the interviews, in note-form, by reviewing the voice recordings. Based on the unstructured note-form summary, a more structured summary was made by categorising the notes into key topics. With the more structured interview data, the differences and similarities of the answers given by the bars are now easily extractable. The refined interview data follow below.

\subsubsection{Control}
\label{sub:Control}
Bars use music to create their image, and music can often be the deciding factor when people are deciding to stay or leave. Therefore all bars expressed, that they want to be in full control of the music, and have the ability to override any request and rearrange the playlist if they feel the need.

\subsubsection{Music Flow}
\label{sub:Music Flow}

All bars want to be able to control the flow of the music playing. That is, they do not for instance want a slow track playing right after a party hit in the middle of a party night or alike. This is currently managed by prepopulating a playlist from which the music system plays tracks from. Guests can at any time request a track to be played, but no guarantees are given whether the requested track is played or not. When a request is given, the bartender must be the judge of whether or not the requested track is matching the current music theme.

\subsubsection{Music Systems and DJ}
\label{sub:differences}
Fabrikken and Dan's Poolhall have no DJ employed so their music system is in use throughout their opening hours, while LA Bar and Newcastle have a dedicated DJ playing at night. One participant uses MiB Pro, while the others use Spotify. A remark was made that MiB Pro did not have as large a selection of tracks as could be desired.

\subsubsection{Concerns with stability}
\label{sub:specific_remarks}

Dan's Poolhall is concerned that streaming music from the internet is risky, because of potential internet failures. A backup solution working offline would be ideal.

\subsubsection{Summary}
\label{sub:summary}

From these interviews, requirements have been gathered based on the participants' responses. These requirements are listed beneath using the MoSCoW method introduced in~\cite{benyon2013designing}. This method provides a structured way of prioritising user requirements. It is important to prioritise all requirements because projects do not have unlimited resources, and because of this have to solve the greatest requirements first.

\subsubsection{Must have}

\begin{itemize}
        \item Ability for the bar to control what music is being played.
        \item The system should always play music, no disruption in playback unless desired.
\end{itemize}

\subsubsection{Should have}

\begin{itemize}
        \item No need for bartenders to judge requests from guests
        \item Ensured continuity in the tracks that are played
\end{itemize}

\subsubsection{Could have}

\begin{itemize}
        \item The ability to take music flow into account
        \item No need for a prepopulated playlist
\end{itemize}

\subsubsection{Want to have}

\begin{itemize}
        \item System works offline
\end{itemize}
