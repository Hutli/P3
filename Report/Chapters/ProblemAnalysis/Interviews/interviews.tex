\label{interviews}

Understanding the people involved in the system is key when doing an analysis, to find their requirements and needs of the system. It is important to get an understanding of current workflows and problems associated with these. One technique towards gaining an understanding of the users, is by interviews. The following section describes how the interviews were conducted and the results that were collected.

\subsection{Procedure}
\label{sub:procedure}
Interviews can be structured, unstructured or anything in between~\cite{benyon2013designing}. Structured interviews are very strict in their form. Every question is pre-made and the interviewer follows the questions strictly. This reduces the effort needed by the interviewer and allows the questions to be well organised and better planned. However this type of interviews should only be used when the answers given by the interviewee are simple and can fall into discrete categories.

On the other end, unstructured interviews allow for a high degree of flexibility. This form of interview is good when knowledge in a particular field is not easily accessible. This flexibility comes with the need for the interviewer to be able to generate questions on the fly and the possibility that these questions are not planned as well and possibly do not cover all aspects.

The interviews described in this section are all semi-structured i.e. most questions are written beforehand but the interviewer is not strictly bound to these and can therefore drill into particular answers given, by following up with related improvised questions. Hence the semi-structured nature of the interviews.

\subsection{Data Collection}
\label{sub:data_collection}
During the interviews the participants were voice recorded and a designated interview helper was taking notes and managing the voice recordings.

\subsection{Interview Analysis}
\label{sub:interview_analysis}
After all the interviews were performed, the group collaboratively created a summary of the interviews, in note-form, by reviewing the voice recordings. Based on the unstructured note-form summary, a more structured summary was made by categorising the notes into key topics. With the more structured interview data, the differences and similarities of the answers given by the participants are now easily extractable. The refined interview data follow each section.

\subsection{Participants}
An important choice in interviewing is who to speak with. There exists two types of people in the interaction this project works with; the bar manager or bartender, who administrates services and protects the interests of the bar, and the guests who visit the bar. The first round of interviews conducted was focused on the administers and can be found in \cref{sub:administersinterviews}. The second round was conducted with the guests at the venue and can be found in \cref{userInterviews}.

\section{Interviews with Administrators}
\label{sub:administersinterviews}
The questions for the administrators were based in the following categories:

\begin{itemize}
  \item Management of bar music systems
  \item Handling requests from the guests
  \item Current and former music systems
  \item Music requirements and the dynamics thereof
\end{itemize}

The full interview guide can be found in \cref{app:interview_guide}.

As this report focuses on the context of bars, the participants would have to have experience working at a bar and using the current music system at the bar. Therefore the sensible choice is to find participants in bars. There are different types of bars: sports bars, bars with DJs at night and bars with a pub atmosphere.

It was chosen to conduct interview with four bars, to get a good representative view of most bars.

\subsection{Control}
\label{sub:Control}
Bars use music to create their image, and music can often be the deciding factor when people are deciding to stay or leave. Therefore all bars expressed, that they want to be in full control of the music, and have the ability to override any request and rearrange the playlist if they feel the need to.

\subsection{Music Flow}
\label{sub:MusicFlow}
All bars want to be able to control the flow of the music. Music is a big part of the experience at a venue and the right music can make guests stay longer. Therefore the administrators desire the music to stay on the target profile and not do sudden changes. For instance, they do not want a slow track playing right after a party hit in the middle of a party night. The administrators want to give the guests power to control the music but not enough to damage the reputation of the bar.

%This is currently managed by prepopulating a playlist from which the music system plays tracks. Guests can at any time request a track to be played, but no guarantees are given whether the requested track is played or not. When a request is given, the bartender must be the judge of whether or not the requested track is matching the current music theme.

\subsection{Music Systems and DJ}
\label{sub:differences}
Two of the bars have no DJ employed so their music system is in use throughout their opening hours, while the two others have a dedicated DJ playing at night. One participant uses MiB Pro\footnote{\url{http://www.beatpro.dk/mib-bag-baren}}, while the others use Spotify\footnote{\url{https://www.spotify.com/dk/}}. A remark was made that MiB Pro did not have as large a selection of tracks as could be desired.

\subsection{Concerns with Stability}
\label{sub:specific_remarks}
It was very important for all bars that the music did not stop playing. It was expressed that this is due to the atmosphere the music created. The music should therefore always play even when nobody has voted for anything and/or there is nothing on the playlist. They also expressed that sometimes the venue most be able to purposely pause or stop the music to for instance give announcements, close the bar or alike. Furthermore one bar owner was concerned with streaming music from the internet, because of potential internet failures. A backup solution working offline would be ideal.

\subsection{Summary}
\label{sub:summary}

From these interviews, requirements have been gathered based on the
participants' responses. The following list of requirements is
prioritised with the most important requirement first.  It is
important to prioritise all requirements because projects do not have
unlimited resources, and because of this have to solve the greatest
requirements first.

\begin{enumerate}
\item Ability for the bar to control what music is being
    played.
\item The system should always play music with no disruption in playback unless desired.
\item Ensured continuity in the tracks that are played.
\item The ability to take music flow into account.
\item System should continue playing even without internet access.
\end{enumerate}
