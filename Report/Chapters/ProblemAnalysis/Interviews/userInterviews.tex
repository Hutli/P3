\subsection{Interview of users}
\label{userInterviews}

There exists two types of end users for the product, the bar manager which administrates the service and the guests at the bar. The guests are allowed to affect the music at the bar, by use of the application. So to understand which type of persons that could be interested in using the application, some interviews of guests were performed.

The interview was conducted in the same semi-structured nature as described in \cref{sub:procedure}. The questions for the interview can be found in \cref{app:interviewguidePotentielleBrugere_guide}. In order to find the right participants for the interview, the interviews were performed with people at representative locations in Aalborg. It was chosen to do interviews at these 4 locations in Aalborg:

\begin{itemize}
    \item \emph{BasisBaren} the biggest bar at the university of Aalborg.
    \item \emph{West End} a bar located slightly away from center of Aalborg.
    \item \emph{Baresso} a coffee shop in the center of Aalborg.
    \item \emph{Studenterhuset} a bar and social meeting place for students of Aalborg university.
\end{itemize}

Six interviews were done with a total of 12 guests, which means some of the interviews was performed on a group of people. All the participants were chosen randomly at the location. 

All the participants that were interview was between 19 and 26, years old. Most of the people said it was a great idea to implement a system which could adjust the music at bars to cater to the guests music preferences. There were some disagreement between the participants in how the system should determine the next song to be played. Some said the guests should chose the next songs, while others said the guests should chose the genre of the music to be played.

Most of participants frequently used their phones, while they were going out. The phone were mostly used for contacting each other while going out, or used to connect with social medias. 
Based on the interviews it was discovered that some of the participants find it annoying if to many people uses their mobiles when they go out, instead of socializing by talking to people at the bar.  

It was discovered that some of the people did not actually listen to the to the songs that were played at the bar, but still all agreed that there has to be music when your going to a bar. Between the participants some of the 19 to 20 years old often requested the bar to play a specific songs. 
