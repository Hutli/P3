\section{Interviews with Guests}
\label{userInterviews}

To understand which types of persons could be interested in using the system and what their requirements would be, some interviews with guests of venues were performed.

The interviews were conducted in the same semi-structured nature, as described in \cref{sub:procedure}. The questions for the interview can be found in \cref{app:interviewguidePotentielleBrugere_guide}. In order to find the right participants for the interview, the interviews were performed with people at representative locations in Aalborg. It was chosen to do interviews at these four locations in Aalborg:

\begin{itemize}
    \item \emph{Venue 1}, one of the biggest bars at Aalborg University
    \item \emph{Venue 2}, a private, well visited, bar located slightly away from the centre of Aalborg
    \item \emph{Venue 3}, a coffee shop in the centre of Aalborg
    \item \emph{Venue 4}, an international bar and social meeting place
\end{itemize}

Six interviews were done with a total of 12 participants, which means
some of the interviews were performed on a group of people. All the
participants were chosen randomly at the location, and were between 19 and 26 years old.

\subsection{Proposals for Guest-Controlled Music}
\label{ProposalsForGuestControlledMusic}
Building on the principle of \citetitle{sorensen2012}~\cite{sorensen2012}, guests were asked what they thought of the concept of a multi-device interaction, where each guest can affect the music at venues, to cater to their own music preferences. Most of the participants thought that it was a great idea. There was some disagreement between the participants in how the system should determine the next track to be played. Some proposed that the guests should choose the next tracks, while others said the guests should choose the genre of the music to be played.

\subsection{Smartphone Usage at Bars}
Most of the participants frequently use their phones, while they are going out. They mostly use their phone to contact each other while going out, or to connect with social media.
Based on the interviews it was discovered that some of the participants found it annoying if too many people use their phones, when they go out, instead of socialising at the bar.

\subsection{Music at Bars}
It was found that some of the people did not actually listen to the tracks that were playing at the bar, but all agreed that there had to be music when going to a bar. Some of the participants often requested specific tracks at the bar.
