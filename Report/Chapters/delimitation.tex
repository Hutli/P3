\chapter{Delimitation}
During the course of this project, a large number of ideas and ambitions were had for the solution, not all of these were possible to implement in the given time span for the project. In this chapter, the functionalities, which were chosen to be delimited, and the reasoning behind the delimitation will be described.

\section{Offline Playback}
Offline playback, meaning that the solution would be functional, even if the server was offline, was an idea which was brought up in the interviews (see \cref{sub:specific_remarks}).
To enable music to be played offline would require a local music catalogue on the server. This would have to be designed, implemented and tested with the same rigour as Spotify, which would likely be very time costly.
Perhaps even more costly, a local music catalogue would most certainly require licences for the tracks, which are available, meaning that the feature could also cost a lot of money. Additionally, a local music catalogue would require a large amount of storage space\footnote{For a comparison, Spotify used a total of 470 terabytes of storage in 2011, according to their lead engineer (http://www.slideshare.net/ricardovice/spotify-p2p-music-streaming)}, meaning that the feature would either have extreme hardware requirements or be very limited in the amount of tracks, which can be accessed.
In the end, offline playback was deemed to be a too extensive feature, which would likely be costly on both time and monetary resources, to realistically be implemented within the scope of this project.

\section{Lockdown}
When the server receives votes it changes the list of votes and
users. These lists are used to select the next track and convert the
temporary votes to permanent votes just before the next track is
played. This could possibly conflict if a user changes his or her vote
while these tasks are performed by the server. An idea was that a lockdown must be placed 10 seconds before the next track is played, where no user can change their vote.

\section{Security}
It was decided to limit the project from taking care of security. This was decided as functionality was prioritised higher than security, mainly because of the limited extent of the use of the system, which this project is concerned with. Securing the solution was deemed too expensive, with regards to time, to be properly implemented and tested in the course of this project.
This means that any individual, with sufficient knowledge of the inner workings of the system, and the format of the requests made to the server, can abuse the system and change data at will.

\subsection{User Authorisation}
\label{par:user_authorization}
User authorisation on the client side would mean the ability to differentiate users. With this information, a lot of features are possible. The application could for example be made to remember the user's votes, in order to smartly suggest tracks.
Implementing user authorisation in a proper way is not simple however. Some discussion on different ideas for the implementation of user authorisation will be described in \cref{sec:UAuth}.
As the biggest benefit of user authorisation lies in the advanced features, which it is a prerequisite of, it was chosen to delimit user authorisation. This was decided since it was not deemed possible to fit the design, implementation and testing of this feature into the timespan of this project, which would in any case not give any real benefits before some more advanced features were implemented.